\documentclass{literature}

\title{Star Formation History}
\subtitle{First year report \& literature review}
\author{Owen J. Turner}
\authoremail{turner@roe.ac.uk}
\supervisor{Dr. Michele Cirasuolo}
\supervisoremail{ciras@roe.ac.uk}
\slugheader{First year report \& literature review}
\slugauthor{Owen J. Turner}
\logo{/Users/owenturner/Documents/PhD/Latex_Test/UoEcrest.pdf}
\abstract{}
\setcounter{secnumdepth}{4}


\begin{document}

\background
\label{background}



%%%%%%%%%%%%%%%%%%%%%%%%%%%%%%%%%%%%%%%%%%%%%%%%%%%%%%%%%%%%%%%%%%%%%%%%%%%%%%%%%%%%%%%%%%%%%%%%%%%%%%%%%%%%%%%%%%%%%%


\section{Meetings with Michele}\label{meetings}
\subsection{21-10-14}\label{meeting_1}
Mainly we discussed Stellar population synthesis models, going through the details of the grid of stellar evolutionary tracks. There are a series of steps: 
\begin{itemize}
\item We know the evolutionary path on the HR diagram with stellar evolution models. Therefore for each mass of star we know how long it will live for 
\item Draw the lightcurve for each individual star to know how much it is emitting as a function of time 
\item Convolve this with the assumed IMF of the galaxy, given as either the Kroupa form \citep{Kroupa_1993}, or the Chabrier form \citep{Chabrier_2003}, which then weights these lightcurves by the total number of object of given mass. 
\item This produces a synthesised lightcurve at a particular snapshot in time for the entire population of stars 
\item Can repeat this procedure at any time to get a set of models of different ages 
\item Things become more complicated when the star formation history and supernovae are included, as essentially the models are functions not just of the IMF and age, but also of the metallicity and star formation history. Supernovae enrich the ISM gas with metals with each generation of star formation, and blow some gas out of the galaxy completely. 
\item Convert to a plot of flux versus wavelength with stellar atmosphere models, which describe the variation form blackbody emission for the stars   
\end{itemize}
That's essentially the picture - there is inherrent uncertainty in the resultant global emission due to the modelling process and much work went on between the Kennicutt review and the update in refining these SPS models. $\chi ^{2}$ minimisation procedures used to select which model has the best fitting parameters to the data observed across as many wavebands as possible (with dust added manually afterwards). Star formation rate inferred from the best fitting model. \\
We also discussed the four major SFR diagnostic methods, these being UV, FIR, Radio and emission lines. To get the relations given by Kennicutt between the UV, FIR flux and SFR, restrict the models to a particular part of the spectrum and find out how much star formation you would require to produce that amount of flux. \\
In sum, we model how much emission is produced across the spectrum, the SED, for a galaxy. These models depend on age, metallicity, redshift, star formation history and dust. We observe the galaxy in as many wavebands as possible and then fit the model to the data with the parameters allowed to vary. If we have enough data points we can break the degeneracy between age and redshift and determine in a $\chi ^{2}$ minimisation and then marginalisation procedure what the best fitting parameters are. Hence we know about the best fitting parameters for that galaxy, one of which is the SFR. Need to think about this more in terms of the ultimate goal being model fitting - the data collection is a means to this end. Next thing to move onto is reading about how the metallicity can be meaured for a galaxy. 
\subsection{28-10-14: Cosmology Calculator Task}\label{meeting_2}
Met to discuss the cosmological distance calculator task. Set the task of computing the luminosity distance, angular diameter distance, volume and age of the universe at a given redshift, given a particular cosmology and hubble parameter. This is done via numerical integration of the equation which defines the comoving distance: 

\begin{equation}
D_{c} = D_{H} \int ^{z} _{0}\frac{dz\prime}{E(z\prime)} 
\end{equation}

Where $E(z) = \sqrt{\Omega _{m}(1 + z)^{3} + \Omega _{k}(1 + z)^{2} + \Omega _{\Lambda}}$. This will be done using trapezoidal methods, the simpson rule, Romberg quadrature and gaussian quadrature. The luminosity distance is absolutely crucial in observational astronomy, as this is the distance which is used when converting between flux and intrinsic luminosity of the source.

\begin{figure}[!htp]
\centering
\includegraphics[width=0.8\textwidth]{cosmo_dist.png}
\caption{\footnotesize{\emph{Plot of the different cosmological distance measures against redshift}}}
\label{fig:cosmo_dist}
\end{figure}

\subsection{4-11-14: Spectra Task}\label{Meeting_3}
Following the cosmological calculator, emission line fluxes can now be turned into the physical properties of galaxies using the luminosity distance. The task is now to go to the SDSS, take a galactic spectra and play around with fitting polynomials to the continuum, gaussian peaks to the emission lines to get the fluxes, the equivalent widths, BPT diagram, redshift etc. The easy step is then coding the computation of the physical quantities after extracting these values. \\\\
\noindent
Michele suggests that coding in a modular way is the best way to do this. e.g. have a module which loads in the fits file in the first place given a list of ID's, then a module which computes the redshift from the plotted spectra, then a module which locates the emission line peaks given the redshift of the galaxy, then a module which computes the physical quantities once the rest has been done. Got an email from Michele on this date listing the SDSS references, data access files and papers to consult whilst attempting to do this. The stress here is definitely on taking the spectra and playing around with it. Try different things, look at how the continuum could be fit by masking out the emission lines, or by using a polynomial of high order. Probably the best thing to do would be to smooth the spectrum using the moving average function defined already, subtract this from the data, fit the continuum, divide the original spectrum by the continuum spectrum and then proceed to do some science from that. \\\\
Also asked Michele some questions about the process of determining metallicity in general. Line fluxes bear some relation to atomic abundance. These relations need to be calibrated theoretically by using codes like cloudy, or empirically by looking in the local universe at the electron temperature and density, which are better resolved there and then extending this to higher redshifts. The reason for expressing the abundance in terms of log(OH) is just as a normalisation factor. 
\subsection{13-11-14: Meeting after spectra task}\label{Meeting_4}
Discussing my initial attempt at the spectra task and taking a look at some of the results. So I think I did an okay job but the continuum fitting isn't quite right, which in turn is affecting the way that the gaussians are being fitted to the emission lines. Instead of blindly fitting a polynomial I should be masking off the emission lines in some way. Michele suggested the local continuum method, whereby in the vicinity of emission lines I estimate the local continuum and subtract this at each stage before fitting the line profile. Also since the gaussian models seem to underestimate the peaks of the emission lines, Michele suggested using a two gaussian fit, to account for broader features at the base of the line profile which are causing the peak of the gaussian to fall too early. Also binning up the continuum to get a better signal to noise from the continuum subtracted data. \\
How to estimate the error on the line fluxes? Extracting the errors on the individual flux points, which comprise the emission line profile. Then for each of these points generate gaussian random variables with mean value of the original datapoint and sigma of the point error. For all of the points which make up the peak, draw this random variable and then fit a gaussian to this and compute the flux, store this in a vector. Do this 100 times to get a distribution of fluxes, the resulting standard deviation of this gaussian is the overall error on the flux. Monte Carlo error estimation. \\ 
Finally Michele suggested that I be given some spectral templates to cross-correlate with these SDSS spectra in order to determine the redshift. Two different techniques for doing this - one applicable in the high signal to noise regime, this being the simple cross correlation function (shift and compute probability, then plot probability against redshift) and another for the low Signal to noise limit where individual probability spikes won't be visible in the cross correlation function. This second method minimises $\chi ^{2}$ but requires an additional fudge factor to accurately explore the two dimensional $\chi ^{2}$ surface. \\ 
Tasks are: 

\begin{itemize}
	\item better explore the continuum with the techniques suggested by Michele 
	\item Continue to read the Kewley and Dopita theory paper 
	\item fit double gaussian line profiles 
	\item continue to play with the data and figure how to mask emission lines etc 
	\item cross correlation functions from different galaxy spectral templates to determine the redshift
	\item Monte Carlo error estimation for the line fluxes
\end{itemize}

An initial plot of the different galaxy templates looks like this: 

\begin{figure}[!htp]
\centering
\includegraphics[width=0.8\textwidth]{spectral_templates.png}
\caption{\footnotesize{\emph{Plot of the spectral flux templates on the same axes}}}
\label{templates}
\end{figure}

Also was looking at Fergus poster in the corridor and a few points stood out. The physical process driving the mass metallicity relation is that galaxies require pristine gas in order to form stars, more massive galaxies drag in more of this from the IGM due to their increased gravitational potential and also hold onto more of their enriched gas for the same reasoning. So it would be expected that the most massive galaxies are also the most metal enriched. \\ 
Also it would appear that Fergus is fitting all of his emission lines simultaneously using a composite model, rather than fitting the lines individually. Should maybe attempt to do this again because this seems to be the way it is done in all of the papers. \\

\subsection{3-12-14: Continuing with spectrum fitting and python}
I decided to collate everything into a class after the visit from Claude. Much better organisational structure and can be easily imported into any program afterwards. Made class objects for both the observed and template spectra - called spectrumFit and templateSpec. Also have the Monte-Carlo flux error, the equivalent width and the equivalent width error working. All of the methods are stored within the spectrumFit class and can be found either locally or on github.  

\begin{figure}[!htp]
\centering
\includegraphics[width=0.8\textwidth]{template_spectrum.png}
\caption{\footnotesize{\emph{Plot of the normalised late type galaxy spectrum}}}
\label{fig:norm_temp}
\end{figure}

\begin{figure}[!htp]
\centering
\includegraphics[width=0.8\textwidth]{fitted_spectrum.png}
\caption{\footnotesize{\emph{Plot of one of the fitted SDSS galaxy spectra}}}
\label{fig:fit_spec}
\end{figure}


%%%%%%%%%%%%%%%%%%%%%%%%%%%%%%%%%%%%%%%%%%%%%%%%%%%%%%%%%%%%%%%%%%%%%%%%%%%%%%%%%%%%%%%%%%%%%%%%%%%%%%%%%%%%%%%%%%%%%

\section{Kennicutt 98: Star Formation in Galaxies along the Hubble sequence}

\subsection{Introduction}
Properties which are affecting the large scale star formation rate within galaxies: 
\begin{itemize}
\item Morphological type 
\item Gas content 
\item bar structure 
\item dynamical environment
\end{itemize}

The first precise diagnostic techniques for measuring the SFRs of galaxies came in the 1970's and 1980's and included integrated emission line fluxes (Cohen 1976, Kennicutt 1983a), near ultraviolet continuum fluxes and IR continuum fluxes. More techniques developed as technology advanced. Interest in the field developed hugely over the following decade, due to two major revelations. The first was the discovery of a population of ultraluminous IR starburst galaxies (ULIRGs) by the Infrared Astronomical Satellite (IRAS) in the mid 1980's, an ubiquitous and extreme phenomenon. The second is the discovery of SFGs at high redshift, $z \sim 3$, allowing for the locally calibrated SFR diagnostics to be applied to distant galaxies and to directly trace the evolution of the SFR density and the Hubble sequence as a function of cosmic lookback time. Galaxies exhibit a huge dynamic range in SFRs, over six orders of magnitude even when normalised per unit area and galaxy mass.

\subsection{Diagnostic Methods}
Since individual young stars are unresolved in all but the closest galaxies, even when using HST, most of the information surrounding the SFR comes from integrated light measurements in the UV and FIR and from nebular recombination lines. Synthesis modelling forms the basis of all the methods. 

\subsubsection{Synthesis Modelling}
Along the Hubble classification fork the spectra of galaxies change in several very noticeable ways. The blue continuum increases broadly, a gradual change in the composite stellar absorption spectrum from K-giant dominated to A star dominated and a drastic change in the strengths of nebular emission lines, particularly H$\alpha$. It is easy to show that the integrated light from stellar populations, although containing contributions from all stellar masses, is dominated from intermediate-mass main sequence stars. As a result the spectra and colors of galaxies fall on a relatively tight sequence, and the spectra of any particular object is dominated by the ratio of early to late-type stars. This makes it possible to use the observed colours to estimate the fraction of young stars and the mean star formation rate of recent times ($10^{8-9}$ years.) \\
The simplest application of this approach would be to assume a linear scaling between the SFR and the continuum emission integrated over a particular blue or ultraviolet bandpass. Although this is a good approximation for young star forming galaxies the assumption breaks down for older galaxies, where a considerable fraction of this continuum emission is emitted by the older stars in the population. The scaling between SFR and continuum luminosity is however a smooth function of the colour of the stellar population, and this can be calibrated using an evolutionary sythesis model. \\
A grid of stellar evolution tracks is used to derive the effective temperatures and bolometric luminosities for various stellar masses as a function of time. So the equations of stellar evolution are coded into a time, mass, age, metallicity, SFH grid, and the differential equations are numerically evaluated at each point so that the values of the effective temperature and bolometric luminosity can be read off for various stellar masses as a function of time. The individual stellar templates are then summed together and weighted by an IMF, Krupa or Chabrier, which has a tail off at the low mass end of the IMF, to synthesize the luminosities, colours, or spectra of single age populations as functions of age. These isochrones can then be added together in linear combination to synthesise the spectrum of colours of a galaxy with an arbitrary star formation history. The four free parameters usually used in these models are the star formation history, galaxy age, metal abundance and IMF. Widely used models of star forming galaxies include those of Bruzual and Charlot  (1993), Bertelli et al. (1994) and Fioc \& Rocca-Volmerange (1997). \\ 
The models generate the SFR per unit mass or per unit luminosity as a function of the colour of the galaxy. The broadband luminosity of the galaxy by itself is a poor tracer of the SFR, as the SFR is shown to vary by more than an order of magnitude over the relevant colour range in Figure 2 of this paper. We need more information about the emission of the galaxy as a function of wavelength. More photometric points, more details about age, metallicity, history. The SFRs generated in this way are relatively imprecise, prone to systematic errors arising from uncertain knowledge of the IMF, metal content, reddening and age. The method should be avoided in cases where these quantities are likely to change systematically across a population of galaxies. \\ 
\subsubsection{Ultraviolet Continuum}   
The above problems can be avoided by observing in a particular wavelength range that is free from the contamination of older stellar populations, and then assuming a linear relationship between the SFR and UV luminosity.The optimal wavelength range is 1250-2500 angstroms, longward of the Lyman alpha forest, but short enough to minimise the contamination from older stellar populations. This wavelength range is inaccessible from the earth for local galaxies (z   \textless 0.5), but can be observed in the redshifted spectra of galaxies with $z \sim 1-5$. The Keck telescope first observed the redshifted UV continuum in a large sample of galaxies with z \textgreater 3. The synthesis models are again used to calibrate the relationship between measured UV continuum and star formation rate. The most important thing coming out of this section is that it's important to apply an SFR calibration method that is appropriate to the population of interest. The real benefit of using the UV continuum method is that it relies upon measuring the photospheric emission of the young stellar population and it can be applied to star forming galaxies over a wide range of redshifts. As a result it is currently the most powerful probe of the cosmological evolution of the SFR \citep{Madau_1996}. The chief drawbacks of this method are the sensitivity to extinction and the form of the IMF.
\subsubsection{Recombination Lines}
As is shown in Figure 1 of this paper, the most dramatic change in the integrated spectrum with galaxy type is a rapid increase in the strengths of the nebular emission lines, which effectively re-emit the integrated stellar luminosity of galaxies shortward of the Lyman limit and so are sensitive to the young massive stellar population. The only stars which produce this flux of ionising photons are the young and massive population, and so the strengths of the H$\alpha$ etc emission lines are directly proportional to the number of young stars and hence the star formation rate. The relationship between SFR and nebular line strength is calibrated using a stellar evolutionary synthesis model. The primary advantages of this method are the direct coupling between the nebular emission and massive star formation rate (the lower mass stellar population simply don't produce high enough energy photons) and the high sensitivity. As before, the limitations of this method are the uncertainties associated with extinction, the IMF and the assumption that all of the star formation is being traced by ionised gas. Estimates of the escape fraction are important when considering how to relate the nebular emission line strength to the SFR. The extinction of photons emitted from ionised gas clouds is the most important source of error, and the line of sight extinction can be estimated by comparing H$\alpha$ fluxes with IR recombination line fluxes. The ionising flux is produced almost exclusively by stars with M \textgreater 10 solar masses, and so the measured SFR is especially sensitive to the choice of IMF. However the H$\alpha$ line equivalent widths and galactic colours are sensitive to the slope of the IMF over a wide range of stellar masses and these can be used to constrain what that slope is. 
\subsubsection{Forbidden Lines}
Due to H-alpha being redshifted out of the optical window at z \textgreater 0.5, there is much interest in calibrating bluer emission lines as quantitative SFR tracers. However these are weaker and more susceptible to stellar absorption. The strongest emission feature in the blue is the OII forbidden line doublet. Forbidden lines are not directly coupled to the ionising luminosity, however OII is well behaved enough that it can be empirically calibrated using H$\alpha$ as a quantitative SFR tracer. The main advantage of using this emission line is to examine the SFR and hence SFH up to higher redshifts and lookback times. Especially useful when used to look at high redshift galaxies, and as a consistency check of SFRs derived in other manners.
\subsubsection{Far-Infrared Continuum}
A significant fraction of the bolometric luminosity of a galaxy is absorbed by interstellar dust and re-emitted in the thermal IR, at wavelengths of roughly 10-300$\mu m$. Since the dust is preferrentially absorbing in the UV, in principle the FIR emission can be a sensitive tracer of the young stellar population and the SFR. The efficacy of the FIR emission as a tracer of SFR, depends upon the contribution of young stars to heating of dust, and on the optical depth of dust in star forming regions. The simplest physical situation where this could apply would be a galaxy in which the UV-optical continuum emission is dominated by the young stellar population, and the dust opacity is high everywhere. Then the strength of the FIR emission is just directly proportional to the SFR, with the measurement being calorimetric. The real physical situation is more complex than this in the disks of normal galaxies. There are two components to the dust spectrum: The First a warm component, coming from the heating of the young stellar population with $\lambda _{peak} = 60\mu m$ and a cooler component ($\lambda _{peak} \textgreater 100\mu m$) arising in more diffuse dust clouds heated by the ISM radiation field. In blue galaxies both components can be the result of the young stellar population, but in red galaxies there are far less young stars and dust heating from the visible spectra of old stars may become very important. As a result of these two components, directly relating the FIR emission to SFR is a controversial topic and the calibration is not straight forward.
\subsection{Disk Star Formation}




%%%%%%%%%%%%%%%%%%%%%%%%%%%%%%%%%%%%%%%%%%%%%%%%%%%%%%%%%%%%%%%%%%%%%%%%%%%%%%%%%%%%%%%%%%%%%%%%%%%%%%%%%%%%%%%%%%%%%%



\section{Kennicutt 2012: Star Formation in the Milky Way and Other Galaxies}
\subsection{Star Formation Rate Diagnostics: The Impact of Multiwavelength Observations}
This section is describing the improvements in calibration and validation of the diagnostic methods for determining SFRs in galaxies. Updated synthesis modelling, IMFs, modelling of dust etc have all contributed towards reducing uncertainties, by up to an order of magnitude in several cases.
\subsubsection{Star Counting and analysis of the colour magnitude diagram}
Having the resolution to count individual stars within galaxies and use the formula given on page 545 to find the SFR. 
\subsubsection{UV Continuum Measurements: The Impact of GALEX}
For a conventional IMF, the peak contribution to the UV flux shortwards of the Lyman-continuum is from stars with several solar masses. Hence measurements of this continuum should in theory directly trace stars formed over the past 10-200Myr. GALEX revolutionised this area, imaging two-thirds of the sky in the FUV(155nm) and the NUV(230nm) channels \citep{Martin_2005}. This was the first space based UV mission and so provided integrated UV fluxes and hence SFR estimates for hundreds of thousands of galaxies. 
\subsubsection{Emission Line Tracers}
For a conventional IMF these lines trace stars with masses greater than 15 solar, with peak contribution from stars in the mass range 30-40 solar. As such, the lines provide a nearly instantaneous measure of the SFR, tracing stars with lifetimes of 3-10Myr. \\ 
Why isn't Lyman-a used instead of H-a? The strength of the former is 8.7 times that of the latter in Case B recombination, making it an attractive tracer in pricinple, but in realistic ISM environments the line is subject to strong quenching from the combination of resonant trapping and the eventual absorption by dust, usually quantified in terms of a Lyman-a escape fraction. 

%%%%%%%%%%%%%%%%%%%%%%%%%%%%%%%%%%%%%%%%%%%%%%%%%%%%%%%%%%%%%%%%%%%%%%%%%%%%%%%%%%%%%%%%%%%%%%%%%%%%%%%%%%%%%%%%%
	
\section{Metallicity Measurement: OIII as an abundance indicator at high redshift}
\subsection{Introduction}
The ratios of the strength of nebular lines are used as a metallicity indicator. Nebular line fluxes reduce dramatically when examining high redshift galaxies, and the wavelengths of these lines are shifted into the NIR, where the sky background is orders of magnitude higher than in the optical. Also, at a given redshift only a subset of the strong lines will fall into an atmospheric window and thus be accessible from the ground. Even at particularly favourable redshifts, such as $z\sim2.3$ which shifts OII, OIII and H$\beta$, and H$\alpha$ and NII respectively into the middle range of the J, H and K bands, it is not possible to record all of these lines in a single exposure. Typically the observation of these lines requires different spectrograph settings, and this can easily introduce additional errors in the relative flux calibration. For these reasons there are practical limitations to the accuracy with which emission line ratios can be measured in high redshift objects.   




%%%%%%%%%%%%%%%%%%%%%%%%%%%%%%%%%%%%%%%%%%%%%%%%%%%%%%%%%%%%%%%%%%%%%%%%%%%%%%%%%%%%%%%%%%%%%%%%%%%%%%%%%%%%%%%%%%%




\section{The Evolution of the mass-metallicity relation at z greater than 3}\label{sec:maiolino metallicity paper}
\subsection{Introduction}
\citep{Maiolino_2008} Aimed at determining the evolution of the mass-metallicity relation at $z > 3$. For an initial sample of 9 star forming galaxies, the gas metallicities are measured by means of optical nebular lines redshifted into the NIR. The galaxy masses are accurately measured using Spitzer/IRAC data, which samples the rest frame near-IR stellar light in these distant galaxies. Previously, since the mass of galaxies has been so hard to measure, researchers have reported correlations between luminosity and metallicity, which is true in the broad sense that more luminous galaxies are also higher metallicity. However the study by Tremonti et al. \citep{Tremonti_2004}, looking at a sample of 53,000 SDSS galaxies, clearly revealed that the primary physical parameter driving the correlation with the gas metallicity is the (stellar) mass of galaxies and not their luminosity. Also there is a difference between the gas metallicity and the stellar metallicity, with similar relationships for both of these quantities. But what is responsible for the mass-metallicity relationship? One possibiltiy is that outflows, generated by starburst winds, eject metal enriched gas into the IGM preferrentially out of low mass galaxies (due to the shallow gravitational potential wells), making their enrichment less efficient than in high mass systems. Alternatively, low mass systems are still at an early evolutionary stage and are still to convert most of their gas into stars, hence they are poorly metal enriched relative to massive galaxies. This is so called `galaxy downsizing', where massive galaxies already formed most of their stars rapidly and at high redshift, whereas the evolution of low mass galaxies extends to low redshifts. A third scenario is that the relation is a result of variations in the IMF high mass cutoff in different star forming environments. These factors have profound impact on the galaxy evolution. Therefore, it is clear that the mass-metallicity relation contains a wealth of information useful to constrain models of galaxy formation and evolution. 
\subsection{The AMAZE program}
In this paper galaxies within the range $3 < z < 3.7$ are imaged using exposure times between 3 and 7.5 hours on source, to collect their spectra using the SINFONI near-IR integral field spectrometer. These observations are used to determine the gas metallicity, the details of this measurement are given in a later section but they rely upon strong line diagnostics and the $H \beta$ and O[III] lines which have been redshifted into the K-band.
\subsection{Gas Metallicity}
Describing the various different ways in which strong line ratios relate to the gas metallicity, and the different methods used to calibrate these ratios. There is generally a problem in using different methods, since they don't produce the same gas metallicity, in fact varying by large amounts. There is no single method which is applicable over the entire range of metallicity observed in high redshift galaxies, and so the authors here rely on two different methods: At low metallicities using the locally calibrated electron temperature method, however at higher metallicities this method tends to saturate and to underestimate significantly the true metallicity, due to temperature fluxuations and gradients, both within individual HII regions and over the whole galaxy. At higher metallicities then, photoionisation models are an alternative method of calibrating strong line ratios. These are all subject to significant uncertainties and possible systematic effects.
\subsubsection{Determination of the gas metallicity}
Describing for 6 different calibrations and line ratios how the gas metallicity is inferred and then making plots of this.
\subsection{Stellar Masses}
Determining the stellar mass and other parameters with a standard $\chi ^{2}$ fitting to templates. Synthetic libraries of templates used, i.e. modelled photometry with different parameters varying. That is why as many photometric points as possible are gathered when carrying out this fit. The BC03 \citep{Bruzual_2003} templates are used when comparing to the results of others, since these are used in previous analyses. However the M05 \citep{Maraston_2005} models are preferred since they include the contribution of TP-AGB stars to the evolution of the system.
\subsection{The mass metallicity relation at high redshift}
Emphasis on cross-calibrating both the metallicity scale and the mass scale, when previous worked have used different IMFs. Different surveys at different redshifts use different strong emission line ratios in order to infer the metallicity. The mismatch between the different calibration scales may introduce artificial evolutionary effects of the mass-metallicity relation. Different datasets are used for the three plots show in Fig. 7, demonstrating the evolution of the mass metallicity relation at different redshifts. Careful steps are taken to ensure that the data are properly calibrated at each stage. For the comparison to the Erb et al. \citep{Erb_2006} z $>$ 2 results, the metallicity is re-determined in each mass bin using the calibration defined in section 5 so to be consistent with the other results. The authors adopt a consistent quadratic fit to the mass/metallicity data points by shifting the quadratic function used to fit the data points in the local universe in mass and metallicity, and determining two free parameters at each redshift. These parameters are listed in table 5 of the paper.
\subsubsection{Aperture effects}
Since galaxies are characterised by metallicity gradients, with metallicity decreasing towards the outer regions, a possible caveat when comparing metallicities at different redshifts is the different aperture projected on the source. In particular at high redshift spectroscopic surveys are likely to cover the whole galaxy, whereas at lower redshift the higher metallicity central region is sampled preferrentially, an effect which may mimic a metallicity evolution. This is expected to be particularly important for local low mass galaxies, where the covering factor can be as low as $20\%$, however when compared to the steep gradient of the mass metallicity relation, this effect certainly would no dominate.
\subsubsection{Selection Effects}
It is important to consider that when comparing the mass-metallicity relation at high and low redshifts, we are comparing different classes of objects that are not necessarily linked from an evolutionary point of view. As a consequence, the evolution of the mass-metallicity relation inferred in this paper should be regarded as the evolution of the mass-metallicity relation of galaxies representative of the density of star formation at each epoch, and not the evolutionary pattern of individual galaxies. This section seeks to confirm that indeed the galaxies in the sample are representative of the SFR at $z = 3.5$. \\
The selection requirement that stars must have a highly reliable spectroscopic redshift may bias the sample towards sources with strong UV continuum or strong Lyman-alpha, hence higher than average SFR. Basically this section is emphasising that LBGs should not vary significantly than other types of galaxies, but this could potentially be a selection bias effect.
\subsection{The evolution of the mass metallicity relation}
The mass metallicity relation evolves more steeply at higher redshift - since more of the age of the universe is encapsulated between the redshift range 0 - 2 this fact becomes apparent. Figure 9 of this paper makes this fact especially clear. Clearly $z = 3.5$ is a major epoch of star formation activity. Also the slope of the mass metallicity relation seems steeper for low mass galaxies than high mass
\subsection{Comparison with models of galaxy evolution}
It is not simple to compare model predictions with observational results. Indeed, theoretical models predict a variety of galaxy populations, spanning a wide range properies, while observations are limited to samples matching the survey selection criteria. There is significant discrepancy between observations and simulations shown in Figure 10. De Rossi et al. \citep{deRossi_2007} suggest that the reason for the discrepancy is the lack of significant SN feedback in their simulations, which would remove metal enriched gas and lower the global gas metallicity. However the observed data are also inconsistent with simulations including SNII and hypernovae by Kobayashi \citep{Kobayashi_2007}. In summary, there are currently no simulations that can satifactorily reproduce the observed mass-metallicity relationship at z $\sim$ 3. The closest match is probably with Governato 
\subsection{Summary and Conclusions}
The results suggest that galaxies at $z > 3$ are assembled from relatively un-evolved sub-galaxies, whose star formation efficiency is low. Most of the chemical evolution must occur once the small galaxies are already assembled into bigger ones. This implies that most of the merging occurs before most of the star formation.

\section{Savaglio: The Gemini Deep Survey - redshift evolution of the mass-metallicity relation}\label{sec:savaglio metallicity paper}
\subsection{Introduction}
The exploration of the chemical enrichment of distant galaxies is carried out using two different methods. The first is the well known strong emission line method, using emission from warm HII regions in integrated galactic spectra, however this becomes significantly challenging at high redshift with the shift of emission lines into the NIR and the observed strength of lines being very weak. The second method is the observation of absorption lines in the neutral interstellar medium of galaxies crossing QSO sight-lines, and gives information on one line of sight in the galaxy.








%%%%%%%%%%%%%%%%%%%%%%%%%%%%%%%%%%%%%%%%%%%%%%%%%%%%%%%%%%%%%%%%%%%%%%%%%%%%%%%%%%%%%%%%%%%%%%%%%%%%%%%%%%%%%%%%%%%%%%%%%




\section{Metallicity Measurement: Hughes Thesis}\label{sec:Hughes thesis}
\subsection{Introduction: Measuring Metallicity}\label{sub:intro_measure}
The chemical composition of galaxies provides a crucial insight into the processes governing galaxy evolution. Star formation episodes convert gas into stars, and heavier elements are then produced via nucleosynthesis in the cores of these stars. These metals and then expelled into the surrounding medium in the later stages of stellar evolution, thus enriching gas that may become fuel for stars in future star formation episodes. Therefore the abundance of heavier elements present in a galaxy, the metallicity, provides an important indicator of the evolutionary history of a galaxy. There are a number of different methods for estimating the metal content of a galaxy, some of which measure the \textbf{stellar} metallicity and some of which measure the \textbf{interstellar gas} metallicity. Mehlert \citep{Mehlert_2002} et al. describe the measurement of stellar photospheric absorption lines and Savaglio \citep{Savaglio_2004} et al. measure interstellar absorption features to trace the stellar metallicity, whereas the abundance of the interstellar gas is estimated via emission for gaseous nebulae. In general it is easier to measure the metallicity of the diffuse interstellar gas, rather than attempting to measure absorption lines from a poorly resolved stellar population. Need to discuss this point with Michele though. This work focusses on obtaining an estimate of galaxy metallicity via the abundance of oxygen in the interstellar gas, and the terms `abundance of oxygen' and `metallicity' will be used interchangeably. \\

\subsubsection{Determining the metallicity}\label{subs:determining the metallicity}
The gas heated by UV photons emitted by young stars becomes ionised, recaptures electrons and begins to re-emit photons. The spectrum of this emission reveals the chemical properties of the IS gas. This section of the thesis gives examples of the most common elements observed in the spectrum. As well as emission via electron recapture, the electrons in heavier elements can be collisionally excited, and since the density of electrons makes collisional de-excitation unlikely, these excited electrons then decay back to the ground state via photo-emission. Hence the `forbidden' or `collisionally excited' lines are also visible in the spectra of these emission nebulae. The word forbidden is misleading, more appropriate would be to say highly unlikely. \\ 
The intensity of emission lines is dependent upon the electron temperature $T_{e}$, the number density $n_{e}$ and the chemical composition of the gas and so obtaining estimates of these quantities is crucial for obtaining a metallicity estimate. How do we do so? With ratios of pairs of particular emission lines. The electron temperature is found form pairs of emission lines from two different levels of a single ion with different excitation energies. An example formula is given in the thesis for this. The electron number density is determined by looking at two different levels with a similar excitation energy but with different radiative transition probabilities. Once the density and temperature are known, the abundance ratio between two ions can be determined from the relative intensity between the measured lines, with a factor of the emission coefficient $j(T_{e}, n_{e})$. This is the direct $T_{e}$ method and is faced with several observational challenges, e.g. the long integration times required to observe the emission lines required to infer $T_{e}$ and $n_{e}$ and the unresolved O[II] and S[II] lines. As a result other observational methods were developed, empirically calibrating metallicity ratios from $T_{e}$ measurements of nebular regions in the local universe. Theoretical calibrations between different emission line ratios and metallicities are also possible by combining stellar population synthesis models with photoionisation models of nebulae. Another approach is to simultaneously fit all the strong emission lines and use theoretical models to generate a probability distribution of metallicities and statistically estimate abundances. Practically the emission from nebulae are observed in the integrated spectra of galaxies. Using these methods the average global abundances of a galaxy can be determined. 

\subsubsection{Calibration Methods}
So as discussed in the previous section, the calibration between the emission line ratios and the oxygen abundances are either determined empirically, theoretically or by using a combination of data and theory. This section describes the 6 different calibration methods used in this study, all of which quote the abundances of oxygen in terms of $12 + log\frac{O}{H}$? Ratios of emission lines seem to correlate with abundance throughout specific intervals - why are they not always relating to the abundance? Overall, all of these methods rely first on going out with your telescope and collecting the spectra of objects for which the gas metallicity is to measured. Once the raw data is reduced in the data analysis pipeline the end result should be a spectra of the relative strengths of emission lines. These ratios give information about the gas metallicity, and various theoretical and empirical calibrations convert the ratios into gas metallicity. The main methods are the $R_{23}$ line ratio, the $\frac{N[II]}{H_{\alpha}}$ ratio and the $\frac{O[III]}{N[II]}$ ratio. But why are these particular line ratios chosen? 

\subsubsection{Discrepancies between Calibrations}
Discrepancies as large as 0.6dex found between empirical and observation calibration methods found in the study of Liang et al. \citep{Liang_2006}. It is crucial to use the same calibration method for any comparison of the M-Z relation because of these discrepancies. In an effort to facilitate comparisons between the results of various samples, KE08 determined conversions allowing metallicities derived with different calibrations to be converted into the same base calibration. The new conversions removed the 0.7 dex systematic discrepancies. So a better estimate for galaxy gas metallicity is found by using lots of different calibration methods, converting to the same base calibration and then averaging the results from each of these.

\subsubsection{Base Conversion}
Describing the details of how to express different calibration methods in the same base calibration. Fitting a polynomial of degree 4 to find the base metallicity - Kewley 2008 \citep{Kewley_2008} is crucial to understanding the conversions between the different calibrations. The conclusion is that the D02 method is the best base calibration to use, which disagrees with the result in the Kewley paper that the M91 calibration is best. The analysis proceeds in the D02 base calibration for the HRS+ sample of galaxies, and in chapter 9 the author studies the M-Z relation.

\subsection{The Mass Metallicity Relationship}
As the stellar mass and metallicity respectively measure the amount of gas converted into stars and the amount of gas converted into metals, the evolutionary stage of a galaxy can be inferred from reliable knowledge of these two quantities. Therefore the M-Z relation provides a valuable tool for studying the chemical evolution of galaxies. However despite mounting observational evidence for the relation, many questions remain regarding the origin, scatter and possibility of environmental dependence. No strong environmental dependence of the M-Z relation reported throughout the paper.







%%%%%%%%%%%%%%%%%%%%%%%%%%%%%%%%%%%%%%%%%%%%%%%%%%%%%%%%%%%%%%%%%%%%%%%%%%%%%%%%%%%%%%%%%%%%%%%%%%%%%%%%%%%%%%%%%%%%%%%%
\section{Kewley and Dopita: Using strong lines to estimate abundances}\label{sec:Kewley_Dopita}
\subsection{Introduction}
Using a combination of stellar population synthesis and photoionisation models to develop a set of ionisation parameter and abundance diagnostics based only on the use of the strong optical emission lines. These optical emission lines are the recombination lines of both Hydrogen and Helium, as well as the collisionally excited lines observed in one or more ionisation states of heavy elements. Oxygen is commonly used as the reference element because it is relatively abundant, emits strong lines in the optical regime, it is observed in several ionisation states, and line ratios of frequently observed lines can provide good temperature and density diagnostics. However the densities of extragalactic HII regions are so low that the density sensitive line ratios are rarely used.  \\ 
One of the major line diagnostic ratios used is $R_{23}$ first proposed by Pagel et al. \citep{Pagel_1979}, the logic for using this ratio is that it provides an estimate of the total cooling due to oxygen, which given that oxygen is one of the principle nebular coolants, should in turn be sensitive to the oxygen abundance. This ratio then has to be calibrated, which isn't an easy thing to do due to the lack of high quality independent data. For low metallicites, the OIII4363 line can be used, but above this detailed theoretical model fits to the data must be used. As a result there are several different documented calibrations of this line ratio. A further drawback of using $R_{23}$ is that it depends also on the ionisation parameter q, defined as: 

\begin{equation}
	q = \frac{S_{H^{0}}}{n}
\end{equation}

A further difficulty in the use of this ratio is that it is double valued in terms of the abundance, i.e. there will be a lower branch and an upper branch abundance estimate from this. When only double valued abundance diagnostics are available, an iterative approach which solves explicitly for the ionisation parameter also helps to resolve the abundance ambiguities, as will be described in this paper. \\

Major advances throughout the late 90's and early 2000's in modelling the emission spectra of starburst galaxies, due to the availability of new data, better stellar evolutionary tracks which include mass loss and overshooting and recent advances in nebular physics. Better photoionisation models which include self-consistent treatment of nebular and dust physics. These are used in conjunction with stellar population synthesis models to synthesise the spectra of galaxies from the UV to the X-ray regime. Throughout this paper, the authors use these models \citep{Dopita_2000}, \citep{Kewley_2001} to simulate the emission line spectra of HII regions and starburst galaxies respectively - the results of the models are then used to develop an optimal scheme for abundance determination based on the range of possible combinations of bright optical or IR emission lines which are likely to be available to the observer. \\

\subsection{Models}
The stellar population synthesis codes PEGASE \citep{Fioc_1997} and STARBURST99 \citep{Leitherer_1999} were used to generate the ionising EUV field. These codes take the Salpeter IMF with appropriate mass range, a range of different metallicities and then generate the galaxy SED at different timesteps by using the equations of stellar evolution applied to a population of stars goverened by that choice of IMF. This EUV field is then input into the photoionisation and shock code \citep{Sutherland_1993}, where the dust physics are explicitly dealt with and the pressure chosen to correspond to typical values in HII regions. In the rest of the paper, figures correspond to a grid of metallicities between 0.05 and 3 times the solar value.

\subsection{Diagnostic diagrams for the ionisation parameter}
Assuming that the metallicity and the shape of the EUV spectrum are defined, the local ionisation state in an HII region is characterised by the local ionisation parameter and all models with similar q will have very similar spectra. Some of the abundance diagnostics rely upon the value of q, and are not useful unless this value is found. The best way to find q, providing the EUV spectrum of the source is reasonably ell constrained, is to take ratios of emission lines of different ionisation stages of the same element. Data points are generated in the models for the ratios of these emission lines and plotted against a grid of values for q. Third order polynomial fits are carried out first for the O[III]/O[II] ratio to give the diagnostic diagram, however this ratio depends also on metallicity. If the S[III]/S[II] ratio is available this provides a very useful diagnostic diagram since the metallicity dependence is much weaker. Practically then, when data is collected the emission line ratios are found and the value of q for this ratio can be read from one of these graphs.

\subsection{Abundance-Sensitive Diagnostic Diagrams}
The heart of the paper. Now the same is done as before but lots of emission line ratios are plotted against metallicity, for different values of q and I can see that some of these ratios are far more sensitive to the value of q than others. Knowledge of the emission line ratio observationally then tells us what the metallicity is. This is calibration of the emission line ratios, however some of the ratios are clearly double valued. Will not write out the descriptions for each of the diagnostic diagrams here, see the paper. 
Main discussions focused around the ranges in metallicity where some ratios are better than others, i.e. where there are strong responses in the ratio to changing metallicity, and why this is the case. Also the sensitivity to reddening when the lines used in the ratios are separated by a wide range in wavelength. Also the combination of different diagnostic diagrams for those which are twin valued to gain an initial `guess' as to what the metallicity is.

\subsection{Comparison with other bright line techniques}
The data used in previous abundance calibrations have been selected in heterogeneous ways, it is therefore important to take care when comparing different abundance diagnostics to account for any biases introduced throughout the data selection stage. Comparing against the previously popular models of M91 \citep{McGaugh_1991}, Z94 \citep{Zaritsky_1994} and C01 \citep{Charlot_2001}. Kewley presents a model flow diagram for optimising the number of metallicity estimates from a given set of data. Lots of figures showing the methods of the papers listed compared with the average metallicity estimates of all the papers.

\subsubsection{Which Diagnostic to use?} 
Refer back to here in the future - although the N[II]/O[II] seems to be by far the best.

\subsection{Optimised Abundance Determination}
The techniques listed here are all individually subject to either systematic or random errors and limited range in applicability. It is possible therefore to derive a combined abundance indicator which uses contributions from the different methods in different metallicity ranges. This paper provides all the necessary calibrated equations for determining the abundance at sensible values of $12 + log(\frac{O}{H})$. 



%%%%%%%%%%%%%%%%%%%%%%%%%%%%%%%%%%%%%%%%%%%%%%%%%%%%%%%%%%%%%%%%%%%%%%%%%%%%%%%%%%%%%%%%%%%%%%%%%%%%%%%%%%%%%%%%%%%%%%%%


\section{Erb: The mass metallicity relation at $z > 2$}\label{sec:erb}
\subsection{Abstract}
\citep{Erb_2006} From a selection of 87 rest-frame UV selected galaxies with mean spectroscopic redshift that is greater than 2, the authors study the correlation between galay metallicity and stellar mass. The stellar masses are determined from SED fitting to 0.3 - 8$\mu$m photometry, the sample is split into six bins in stellar mass, and six composite H$\alpha$ + N[II] spectra are constructed from all of the objects in each bin.The oxygen abundance is estimated in each bin from the mean N[II] / $H\alpha$ ratio and find a monotonic increase in metallicity with increasing stellar mass. The conclusion is that the mass metallicity relation at high redshift is driven by the increase in metallicity as the gas fraction decreases through star formation and is likely modulated by metal loss from strong outflows in galaxies of all masses.

\subsection{Introduction}
Any first year review report should start off with a discussion of the first measures of the mass metallicity relation, \citep{Lequeux_1979}, and how this subsequently progressed onto discussions of the easier to measure luminosity metallicity relation, e.g. \citep{Zaritsky_1994}, \citep{Skillman_1989} and \citep{Salzer_2005}. It has long been recognised that a correlation between stellar mass and gas-phase metallicity is a natural consequence of the conversion of gas into stars in a closed system. The yield is defined as the mass of metals produced and ejected by star formation, in units of the mass that remains locked in long-lived stars and remnants.

\subsection{Mass and Metallicity Measurements}
Mass was measured by fitting model spectral energy distributions to the multiwaveband photometry, using the procedure described in detail in Shapley \citep{Shapley_2005} and Erb \citep{Erb_2006a} which uses the BCO3 \citep{Bruzual_2003} stellar population synthesis models and the Calzetti extinction law \citep{Calzetti_2000}, with a variety of ages and extinctions to match the observed photometry. The best fit models yield the stellar mass and the SFR. The Stellar masses are computed using a Chabrier IMF \citep{Chabrier_2003}, which results in masses and SFRs 1.8 times smaller than those computed using a Salpeter IMF. Also the Stellar mass is the integral of the SFR over all times, and so is the total mass of stars formed to that date, rather than the current mass in loving stars. For that choice of IMF the current living mass in stars is between 10-40$\%$
lower, depending on the age of the galaxy in question. As discussed by Papovich et al. \citep{Papovich_2001}, one downfall of this type of modelling, i.e. fitting the observed photometry with SPS models and dust extinction laws, is that older stellar populations may be masked by young starbursts, leading to underestimates of the total stellar mass. This is shown by the authors to not be the case statistically, although it can't be ruled out in all cases. \\ 
The metalicities are measured by dividing the sample into six mass bins, with 14 or 15 galaxies in each of these. Now, the most direct way to determine the abundances of metals from the observed emission-line fluxes in HII regions is through the measurement of the electron temperature $T_{e}$. As the metallicity of the gas increases, the cooling through metal emission lines also increases, resulting in a decrease in $T_{e}$. The ratio of the auroral (transition between the second lowest excited state to the first lowest) and nebular (transition between the first excited and ground state) emission lines of the same ion is extremely sensitive to electron temperature and so measuring these lines has been the preferred method for estimating the abundance of metals in HII regions. However the auroral lines, particularly the widely used OII4363 line, become extremely weak at metallicities above 0.5 solar, and undetectable at all in the low S/N spectra of distant galaxies. As a result, people resort to empirical `strong line' methods, which are based on the ratios of collisionally excited forbidden lines to hydrogen recombination lines. \\ 
As usual, the calibrations are subject to significant biases and base calibration discrepancies - so absolute values of metal abundances are still quite uncertain. However, fortunately for this paper relative abundances of similar objects determined using the same method are more reliable. \\ 
The available data at these redshifts limits the options for abundance determination. Have to think about how much observing time would be required to collect the data required for certain ratios. As a result the NII/H-alpha ratio was used for the majority of galaxies in this sample. This ratio was first suggested by Storchi-Bergmann in 1994 \citep{Storchi_1994} and was furthered refined by Pagel and Pettini in 2004 \citep{Pettini_2004}. Using the method described in P04: 

\begin{equation}
	12 + log(\frac{O}{H}) = 8.90 + 0.57 \times N2
\end{equation}

A further problem of the N2 method is that this emission line becomes weak at low metallicity - and is difficult to detect in the spectra of individual objects. In an attempt to overcome this problem, the authors construct the composite spectrum from binning the objects by stellar mass - which increases S/N and averages over metallicity and mass to reduce the errors on both of these quantities. \\ 

Empirically the ratio is measured by first fitting the H-alpha emission line to find the flux, central wavelength and width and then constraining the NII line to have the same width and a central wavelength fixed by the H-alpha position. The rms of the spectrum between emission lines is used to determine the typical noise in each spectrum; because the galaxies are at different redshifts the systematic effects of the night sky lines are minimised in the composite spectrum. This procedure allows the metallicity and metallicity error to be computed easily. 

\subsection{Mass-Metallicity relation}
Clear and unambiguous trend - higher stellar mass and higher metallicities. The origin of this relationship is investigated later in the paper. 
\subsubsection{Composite Ultraviolet Spectra}
These are said to rich in stellar absorption features that provide abundance diagnostics for the young stellar populations. The difficulty is that these are low contrast features, usually requiring data of higher quality than can be obtained with current instrumentation. Nevertheless it is worthwhile to examine whether the rest frame UV spectra of the galaxies under question are consistent with the abundance trend revealed in Figure 3. \\
Two UV spectra are constructed using coarser mass binning than before to improve the S/N so that the spectral features can be clearly distinguished; these are plotted in Fig.4 normalised to the stellar continuum following the procedure described by Rix et al. \citep{Rix_2004} LOOKS LIKE A GOOD PAPER TO READ. Also reproduced synthetic spectra generated with the Starburst99 code \citep{Leitherer_1999} for the standard case of continuous star formation and salpeter IMF. Standard Stellar libraries available using Starburst99, one for the MW galaxy and one for the MC. The higher metallicity library provides a plausible fit to the higher mass bin UV continuum. The authors don't attempt a quantitative comparison - believing that this isn't warranted in the current situation. However at a qualitative level, they conclude that the higher mass bin has roughly solar metallicity and the lower mass bin has roughly MC metallicity - broadly consistent with the values found from the NII/H-alpha ratios. There are other obvious differences in the composite UV spectra, most notably the strengths of the Ly-alpha and CII emission lines. The CII doublet shows a strong inverse metallicity dependence. As metallicity increases, the temp of HII regions decreases, as do the relative populations of the collisionally excited levels from which these emission lines originate. 
\subsubsection{Luminosity Metallicity relationship}
There is clearly a much weaker relationship between luminosity and metallicity than between mass and metallicity. The trend is not monotonic and it is not statistically significant. There is clearly much more physical meaning in a mass metallicity relationship - there is large variation in the optical mass to light ratio at high redshift. A corollary of this is that the observed local luminosity metallicity relationship is a consequence of the tight relationship between mass and luminosity at local redshifts. 
\subsection{The Origin of the mass metallicity relationship}
A correlation between the gas phase metallicity and stellar mass can plausibly be explained by either the tendency for lower mass galaxies to have higher gas fractions \citep{McGaugh_1997}, \citep{Bell_2000} and thus be less enriched, or by the preferential loss of metals from galaxies with shallow potential wells by galactic-scale winds. With the relevant information on the star, gas and metal content of the galaxy, the two effects can be differentiated. \\ 
\textbf{Simple closed box model of chemical evolution}
STILL NEED TO CONTINUE READING FROM HERE ONWARDS. HARDER STUFF. Hello  

%%%%%%%%%%%%%%%%%%%%%%%%%%%%%%%%%%%%%%%%%%%%%%%%%%%%%%%%%%%%%%%%%%%%%%%%%%%%%%%%%%%%%%%%%%%%%%%%%%%%%%%%%%
\section{Tremonti: The origin of the mass-metallicity relation from 53,000 SDSS galaxies}\label{sec:tremonti_2004}

\subsection{Introduction}
Getting to the stage where a lot of this has already been read, so will not take as detailed notes anymore. Just things that seem different to what has been read before. Although there is a really nice opening paragraph here which I'll quote: 
`The stellar mass and metallicity of galaxies are two of the most fundamental physical properties of galaxies. Both are metrics of the galaxy evolutionary process, the former reflecting the amount of gas locked up into stars, and the latter reflecting the reprocessing of gas by stars and any exchange of gas between the galaxy and its environment. Understanding how these quantities evolve with time and in relation to one another is central to understanding the physical processes that govern the efficiency and timing of star formation in galaxies.' \\ 

Feedback has been regarded as an important part of galaxy evolution since the 70's - however it is a complex hydrodynamical phenomenon, with the majority of the energy and newly synthesised metals existing in the hot and hard to observe coronal phases. This complexity has prevented the development of physically accurate prescriptions for incorporating feedback into semi-analytical and numerical simulations of galaxy formation. \\

The development of more sophisticated models for stellar populations in the early 2000's, e.g. \citep{Bruzual_2003}, has resulted in major advanced in our ability to derive physical properties from observables. BY FITTING THE OBSERVED DATA WITH THE MODELS AND FINDING OUT WHAT THE PARAMETERS ARE.
\subsection{SDSS Data}
\subsubsection{Emission Line Measurement}
As expected the continuum is the most complicated part of this analysis and requires work to account for the absorption features as well as the emission features. In order to maintain speed and flexibility, the SDSS spectroscopic pipeline performs a very simple estimate of the stellar continuum using a sliding median average. While this is generally adequate for strong emission lines, a more sophisticated treatment of the continuum is required to recover weak features and to properly account for the stellar Balmer absorption, which can reach equivalent widths of 5A. To account for this the authors developed a special code which makes use of the BC03 \citep{Bruzual_2003} stellar population synthesis models to build a library of template spectra of single stellar populations. The templates include models of different ages and metallicities, and for each galaxy the templates are transformed to the appropriate redshift and velocity dispersion to match the observed data. Then perform a nonnegative least squares fit to determine which is the most appropriate template. Since the authors are particularly interested in the weak spectral features they adopt a special strategy: all of the emission lines are fitted simultaneously with gaussians, requiring the balmer lines have the same width and velocity offset, and also that all of the forbidden lines have the same width and velocity offset. The virtue of these constraints is that they minimise the number of free parameters and effectively allows the stronger lines to be used to help constrain the weaker ones. Extensive by-eye observation suggests that our continuum and line fitting methods work well. This is a really good idea - would it be possible to implement this with lmfit? The weaker lines are always the hardest to measure accurately, so by using the strong lines to constrain their widths would get much fewer free parameters and also more reliable fits. 

\subsubsection{The Galaxy sample}
Impose a redshift cut of $0.005 < z < 0.25$ and that the fibre observes more than 10$\%$ of the total galaxy light. Also need the $H\alpha, H\beta and NII$ lines to be detected at greater than $5\sigma$, so would expect particularly high resolution in the resulting spectra, which does indeed appear to be the case. Also require that the parameters used for galaxy mass estimation have small errors, AGN to be excluded using the traditional BPT diagram and select the star forming galaxies using the formulation of Kauffman et al. \citep{Kauffman_2003}. Result is the trimming of 211000 galaxies down to the 53,400 used in the subsequent analysis. 

\subsection{The Physical Properties of galaxies}
\subsubsection{Measuring Metallicity}
From Oxygen abundance as usual - this is the canonical `metal' used for ISM studies, as it is the most abundant, is only weakly depleted onto dust grains and also displays strong lines in the optical. There are also several advantages to inferring metallicity from nebular lines rather than from stellar absorption features via Lick indices: firstly the S/N in emission lines can greatly exceed that of the continuum. Secondly, nebular abundance is free from the uncertainties due to age and $\alpha$-element enhancement that plague the interpretation of absorption line indices. Thirdly, nebular metallicities are easier to interpret in the context of chemical evolution models because they represent the present-day metal abundance rather than the luminosity-weighted average or previous stellar generations. The disadvantage of nebular line abundance methods is that the analysis is limited to galaxies with ongoing star formation. Strong line abundance determination methods were pioneered by Alloin et al. and Pagel \citep{Alloin_1979}, \citep{Pagel_1979}. Here the metallicities are measured in a more refined statistical way, using the model of Charlot and Longhetti \citep{Charlot_2001} to compute the likelihood distribution of the metallicity - the median of which gives the metallicity and the width of which gives the error on the metallicity. \\ 
An issue which remains is whether spatially averaged spectra can produce truly representative spectra in the presence of radial variations in temperature, ionisation, metallicity and dust extinction \citep{Kobulnicky_1999} 
\subsubsection{Measuring Stellar Mass}
The stellar mass of a galaxy cannot be inferred directly from its optical luminosity as the stellar mass to light ratio depends strongly on the galaxy's star formation history and metallicity. Slightly bizarre Bayesian method of assigning stellar masses, by looking at the M/L ratios for galaxies selected from a grid of realisations with different ages, metallicities and star formation histories. \\ 
\subsection{The Luminosity Metallicity relationship}
Use kcorrect \citep{Blanton_2003} to measure fixed frame galactic magnitudes for comparisons to the results of other surveys, which in the pass have primarily focussed on the luminosity metallicity relationship due to the relative difficulty of measuring stellar mass. Using a linear least squares approach to fit the data as is typical for this relationship. Using the bisector method, which takes the two lines Y v X and X v Y and plots the bisector. The relationship between mass and metallicity found in this work is: 

\begin{equation}
 	12 + log(\frac{O}{H}) = -0.185M_{B} + 5.238
 \end{equation} 

which is intermediate between the results found by Kobulnicky and Zaritsky \citep{Kobulnicky_1999} for local irregulars and spirals and by Melbourne and Salzer. 
\subsection{The mass metallicity relation}
The principal difference between the mass-metallicity and luminosity-metallicity relations is the more pronounced turnover seen at high metallicity when mass is used as the independent variable.
\subsection{The Origin of the mass metallicity relation}
Again either the tendency of lower mass galaxies to have higher gas fractions and thus be less enriched, or for lower galaxies to have shallower potential well and thus be subject to the loss of metals through galactic scale winds. It is not a priori clear whether this scale is one of enrichment or of depletion. For example if over a hubble time, more massive galaxies form fractionally more stars than their less massive counterparts the observed mass-metallicity relation represents a sequence in astration. However, if galaxies form similar fractions of stars, then the relation could imply that metals are selectively lost from galaxies with small potential wells via galactic winds. Recent observations have confirmed that gas mass fractions decrease with increasing stellar mass \citep{McGaugh_1997}, \citep{Bell_2000}, \citep{Boselli_2001}, which is a trend which seems to suggest that low mass galaxies are unenriched rather than depleted. However, observations of starbursts have revealed the nearly ubiquitous presence of galactic winds \citep{Heckman_2002}, while studies of X-ray bright clusters have demonstrated the presence of copious metals in the intracluster medium \citep{Gibson_1997} and absorption line studies have revealed the presence of metals in the intergalactic medium \citep{Ellison_2000}. \\ 
Given the existence of these contradictory pieces of information, anecdotal arguments are of little use - one needs a direct test of the origin of the mass-metallicity relation. This is from chemical evolution models, for which one needs to know the gas mass fraction. This is found by invoking another well known empirical correlation, the Schmidt star formation law \citep{Kennicutt_1998a}, which relates the star formation surface density to the gas surface density. The Schmidt law is inverted to find the gas surface density, the stellar surface density is found using the spectroscopically defined mass to light ratio and the gas mass fraction is then defined as: 

\begin{equation}
	\mu _{gas} = \frac{\Sigma _{gas}}{\Sigma _{gas} + \Sigma _{star}}
\end{equation}
The yield of the galaxies can then be found. The yield is simply the mass of metals produced and ejected by star formation per unit time. The authors then plot the yield as a function of total baryonic mass for all of the galaxies in the survey, finding that baryonic mass and effective yield are highly correlated. Also by the inclusion of the other datasets that galaxies clearly do not evolve as closed boxes and that inflows and outflows are important throughout the evolutionary cycle. The authors then argue that the relationship between baryonic mass and effective yield is the consequence of galactic winds expelling material. 
\subsection{Sources of Systematic Error}
Aperture effects again playing a big role potentially, due to the gradients of physical properties within galaxies. Surveys like the SDSS have strong selection effects due to being magnitude limited. This is unlikely to affect the relationship between mass and metallicity, but may overestimate the metallicites by only considering nuclear abundances, or at least only considering abundances with on average 24$\%$ of the aperture enclosed. \\ 
Another source of error is from the gas mass derived from the H$\alpha$ measurement, although there is impressive correlation in the Schmidt law over several orders of magnitude there is a lot of scatter in the range of measurements relevant to this paper, and Wong and Blitz \citep{Wong_2002} showed that you get a steeper law should you apply a radially dependent attenuation correction to the H$\alpha$ luminosity. A more fundamental question as well is how the products of star formation are distributed throughout the galaxy; in other words, how well mixed is the ISM. The mere existence of radial metallicity gradients suggests that the mixing timescales are generally long. The general result coming from this paper surrounding the origin of the mass metallicity relation arising from the preferential loss of material from galaxies with lower potential wells appears to be very robust. 
\subsection{Summary and discussion}
The most simple and observationally supported conclusion is that galactic scale winds preferentially expel metals from galaxies with shallower potential wells. Whole host of papers given in this section to back up these conclusions. However comparatively little is known about the eventual fate of this material; does it escape the galaxy completely or does it cool and then rain back down onto the galactic disk. The authors also rule out the inflow of pristine gas as being responsible for the low yields observed, as they do not have a significant impact on the yield of galaxies. It seems likely that the blowout events are associated with starburst activity, and this in turn would suggest that the star formation history of most galaxies is bursty rather than continuous. Galactic winds are ubiquitous and extremely effective at removing metals from galaxies. The results imply that metallicity is not a simple model of galaxy evolution, because metals can escape galactic potential wells. 



%%%%%%%%%%%%%%%%%%%%%%%%%%%%%%%%%%%%%%%%%%%%%%%%%%%%%%%%%%%%%%%%%%%%%%%%%%%%%%%%%%%%%%%%%%%%%%%%%%%%%%%%%%

\section{The Dust Content and Opacity of actively star forming galaxies}\label{sec:Calzetti law}
\subsection{Introduction}
The FIR SEDs of galaxies provide insight into the energetic processes taking place within the sources. Emission beyond a few microns from quiescent or star forming galaxies is dominated by dust re-radiating the stellar energy absorbed at the UV/optical wavelengths. Although dust emission is complex, simple two component models have provided good descriptions of the UV extinction and FIR emission in our own and other similar galaxies. The first component consists of very small grains and large molecules, size $<100$ angstroms, and accounts for the characteristics of the mid-IR emission at $\lambda \sim 40-50\mu m$. This component is heated by the single photon absorption process to temperatures between 100-1000K and is not in thermal equilibrium with its environment. The second component is comprised of much larger grains, $> 100A$, in thermal equilibrium with their environment and accounts for nearly all of the emission longwards of 80 microns. The largest data collection of FIR galaxies can be found in IRAS, however the wavelength coverage is between 8 and 120 microns, IRAS observations alone are not enough to classify the emission from dust cooler than 30K. The point is that cooler dust is radiating most of its energy at longer wavelengths, IRAS isn't sensitive at these wavelengths and so to properly classify the FIR SED, need complementary data. This comes in the form of millimeter and sub-millimeter observations, which are however potentially sensitive to radio source contamination. The infrared space observatory (ISO) is sensitive up until 240 microns and so is useful for classifying the FIR dust emission from grains at temperatures less than 15K. ISO finds that a substantial portion of the dust content of galaxies comes in the form of this cool dust. \\
The focus of this paper is on actively star forming galaxies because of the central role they play in interpreting high redshift galaxies. Some of the characteristics of high redshift Lyman break galaxies are similar to those of the central regions of local starburst galaxies. Such characteristics include the star formation rates per unit area, the shape of the stellar continuum and the shape of the absorption features in the rest-frame UV spectra, and the ionising/UV continuum. photon ratios. 

\subsection{Analysis and Results}
\subsubsection{Modelling the emission from large dust grains}
The practicalities - through the combination of ISO and IRAS data, the dust emission has been directly measured in the wavelength range 8-240 for this sample of 8 galaxies with redshift less than 0.03. Once all of the complicated data analysis and reduction in done this leaves the opportunity to effectively model the dust emission longward of the 240 point, thus extrapolating the dust emission at these wavelengths. Modelling of the dust emission shortwards of 40 microns is not attempted, because of its complex nature. \\ 
A range of parameters is explored for the fitting the four data points with $\lambda > 40$: the emission is modelled with a single and with a combination of two modified planck functions. The parameters are explored using $\chi ^{2}$ minimisation, and the fit is well constrained since the number of free parameters is less than the number of independent data points - these being two and four for the single and double modified planck functions. Combination of which gives the best fit in the authors results. 

\subsection{Dust Masses and Gas to Dust ratios}
There are standard recipes for computing the dust masses, found in Hildebrand 1983 \citep{Hildebrand_1983} and Young 1989 \citep{Young_1989}.  







%%%%%%%%%%%%%%%%%%%%%%%%%%%%%%%%%%%%%%%%%%%%%%%%%%%%%%%%%%%%%%%%%%%%%%%%%%%%%%%%%%%%%%%%%%%%%%%%%%%%%%%%%%

\section{Distance Measures in Cosmology}\label{sec:cosmology_distance}
\subsection{Cosmographic Parameters}
Relevant to distance in the parameter $D_{H}$ which is known as the hubble distance, given simply by the speed of light multiplied by the hubble time (inverse hubble constant). The other cosmological parameters, $\Omega_{M}, \Omega_{\Lambda} and \Omega_{k}$ determine the mass density of the universe and hence the time evolution of the metric and the cosmological distance to points at certain redshift.
\subsection{Redshift}\label{subs:redshift}
The redshift of an object is the fractional doppler shift in its spectral lines caused by radial motion. Redshift is directly observable and radial velocity is not; these notes concentrate on observables. There is a difference between the observed redshift and cosmological redshift of an object, caused by the peculiar motion of that object. Cosmological redshift is taken to mean the part of the redshift which is due purely to the expansion of the universe. The redshift values used from here on out will solely relate to this cosmological redshift, which is a far more useful quantity. The two values are related using the relation: 
\begin{equation}
v_{pec} = \frac c{z_{obs} - z_{cos}}{(1 + z)}	
\end{equation}
For small distances the different measures $D_{L}, D_{A}$ etc converge and the simple Hubble approximation can be used: $v_{rec} = H_{0}D$. However this is only true for the very local universe and is not true in general. Generally, the redshift of an object is related to the scale factor via $1 + z = \frac{a(t_{o})}{a(t_{e})}$ where $a(t_{o})$ is the size of the universe at the time the light was observed.
\subsection{Comoving Distance}\label{subs:comoving_distance}
A small comoving distance between two nearby objects in the universe is the distance between them which remains constant with epoch if the two objects are moving together with the Hubble flow. The total comoving distance to an object at redshift z is given by integrating these infinitesimal constributions from objects close to each other, i.e. by the equation in section \ref{meeting_2}. The line-of-sight comoving distance between two nearby events (ie, close in redshift or distance) is the distance which we would measure locally between the events today if those two points were locked into the Hubble flow. It is also in many ways the fundamental distance measure, as all other measures are derived quite simply in terms of this. 
\subsection{Transverse Comoving Distance}\label{subs:transverse_comoving_distance}
This is denoted by $D_{M}$ and is related to the comoving distance by the formulae presented in the Hogg paper. The two quantities are analogous in a flat universe, but when spatial curvature in introduced there is either a sin or sinh function. 
\subsection{Angular Diameter Distance}\label{subs:Angular Diameter distance}
The angular diameter distance DA is defined as the ratio of an object’s physical transverse size to its angular size (in radians). This distance measure doesn't increase indefinitely and turns over at $z \sim 1$, where objects of fixed physical size at higher redshift than this appear larger.
\subsection{Luminosity Distance}\label{subs:Luminosity distance}
The luminosity distance $D_{L}$ is defined by the relationship between bolometric (ie, integrated over all frequencies) flux S and bolometric luminosity L:
\begin{equation}
	D_{l} = \sqrt \frac{L}{4\pi S}
\end{equation}
And after some calculation it can be shown that this is related to the transverse comoving distance, which will be computed first by the numerical integral, by $D_{L} = (1 + z)D_{M}$. If the concern is not with bolometric quantities but rather with differential flux $S_{\nu}$ and luminosity $L_{\nu}$ , as is usually the case in astronomy, then a correction, the k-correction, must be applied to the flux or luminosity because the redshifted object is emitting flux in a different band than that in which you are observing. The remainder of this section then describes in detail the definition of the k-correction.

\subsection{Comoving Volume}\label{subs:Comoving Volume}
The comoving volume $V_{C}$ is the volume measure in which number densities of non-evolving objects locked into Hubble flow are constant with redshift.

\subsection{Lookback time and age}\label{subs:lookback time}
Hello

%%%%%%%%%%%%%%%%%%%%%%%%%%%%%%%%%%%%%%%%%%%%%%%%%%%%%%%%%%%%%%%%%%%%%%%%%%%%%%%%%%%%%%%%%%%%%%%%%%%%%%%%%%%%%%%%


\section{Brinchmann Physical Properties}\label{sec:Brinchmann_properties}
\subsection{Introduction}
Based on the results of previous studies, a picture has emerged whereby more massive galaxies undergo a larger fraction of their star formation at early times than less massive ones. This has often been said to present a challenge to existing models of galaxy formation. Initial studies with relatively small galaxy samples, but with the advent of large fibre-fed surveys it is now possible to dramatically increase the size of the galaxy sample. This work is aimed at probing the SFH history of the universe as a whole, as there are wide discrepancies between the evolution of the SFR above $z = 3$. One of the key questions is what physical parameters drive the changes in the SFR in individual galaxies. It is therefore of crucial importance to investigate empirically what global quantities correlate strongly with star formation activity. Fitting all strong emission lines simultaneously. 



%%%%%%%%%%%%%%%%%%%%%%%%%%%%%%%%%%%%%%%%%%%%%%%%%%%%%%%%%%%%%%%%%%%%%%%%%%%%%%%%%%%%%%%%%%%%%%%%%%%%%%%%%%%%%%%%%

\review
\label{background-review}

\progress
\label{progress}

\summary
\label{summary}

%%%%%%%%%%%%%%%%%%%%%%%%%%%%%%%%%%%%%%%%%%%%%%%%%%%%%%%%%%%%%%%%%%%%%%%%%%%%%%%%%%%%%%%%%%%%%%%%%%%%%%%%%%%%%%%%%%

\section{Definitions}
\begin{itemize}
\item \underline{Dynamic Range}: The ratio of the largest to the smallest possible values of a changeable quantity.
\item \underline{Synthesis}: The combination of components or elements to combine a collective whole. e.g. the evolutionary synthesis of a population of stars.
\item \underline{efficacy}: The ability to produce the desired or intended result
\item \underline{calorimetric measure}: Young stellar population transferring heat to the dust particles, which will emit at a certain peak wavelength
\item \underline{bolometric luminosity}: The luminosity of an object after taking into account all frequencies of electromagnetic waves and extinction and absorption in the Earth's atmosphere.
\item \underline{coeval}: Having the same age or date of origin
\item \underline{Interloper}: Becoming involved in a situation where it is considered not to belong	
\item \underline{abscissas}: In the context of mathematics, the abscissa is the perpendicular distance of a point from the y-axis. The distance parallel to the y-axis is called the ordinate
\item \underline{Gaseous Nebulae}: Gaseous nebulae are optically visible clouds of gas present in our Galaxy and in other galaxies
\item \underline{Forbidden Line}: Emission lines observed in the astronomical spectra of gaseous nebulae, but not from the spectra of the same elements in a laboratory on Earth. The term forbidden is misleading, more appropriate would be highly unlikely. The gas cannot be sufficiently rarefied on Earth, and instead of transitioning from a higher energy level to a lower one via photo-emission, the electrons are many orders of magnitude more likely to do so via collisional de-excitation either with another electron or with the walls of a container.
\item \underline{Lyman alpha Forest}: Observed in the spectra of bright distant objects. The Lyman alpha photons emitted by the object are absorbed by clouds of intergalactic neutral hydrogen between us and the object. Each absorption, at different degrees of redshift leaves its fingerprint in the spectrum. Used to diagnose the frequency and density of clouds of intergalactic neutral hydrogen
\item \underline{Lyman Break Galaxy}: Galaxies observed in at least one filter shortwards and longwards of the Lyman limit show this break in their emission. Colour selection technique whereby high redshift galaxies, or galaxies at a particular redshift, can be picked out. This feature is present in star forming galaxies, where light shortwards of the Lyman limit is almost completely absorbed by dust and nebular emission clouds around star forming regions - so the galaxy exhibits a `break' in its SED at the shifted location of the 91.2nm Lyman limit.
\item \underline{Auroral Line}: Emission between the second lowest and lowest excited states 
\item \underline{Nebular Line}: Emission between the first excited and ground states. The ratio of the auroral and nebular emission lines is extremely sensitive to electron temperature and is the preferred method of estimating metal abundances in HII regions. However the auroral line becomes very weak at metallicites above 0.5 solar, and so it difficult to measure in practice. As a result people resort to calibrated ratios of strong, forbidden emission lines.  
\end{itemize}


%%%%%%%%%%%%%%%%%%%%%%%%%%%%%%%%%%%%%%%%%%%%%%%%%%%%%%%%%%%%%%%%%%%%%%%%%%%%%%%%%%%%%%%%%%%%%%%%%%%%%%%%%%%%%%%%%%

\section{Questions}
\subsection{SFR diagnostics}
\begin{itemize}
\item Is there a reason why initially it would be assumed that high redshift galaxies would not be star forming? 
\item What does it mean a gradual change in the stellar absorption spectrum, from K-giant dominated to A star dominated? 
\item Page 193: `A grid of Stellar evolution tracks...' what does that mean? Also `Converted into broadband luminosities (or spectra) using stellar atmosphere models or spectral libraries' what does that mean? 
\item The article refers to sythesis models in the context of single age populations. What is the approach for populations of stars which are not single age? 
\item Why do the evolutionary synthesis models give relations between the sSFR and the integrated colour of the galaxy? Why choose the integrated colour as the x-axis of the plot? 
\item Confirming the meaning of the use of dex. e.g. differing over a full range of $\sim 0.3$dex.
\item The best determinations of the extinction are based on two component radidative transfer models...
\item The meaning of `Nebular' emission lines, and how are they re-emitting the light which is `shortward' of the Lyman limit. And only young massive stars are emitting these lines? What is the connection with a flux of ionising photons? 
\item Importance of the escape fraction of ionising photons for the measurement of SFR from nebular emission lines.
\item The SFR vs $L_{FIR}$ conversion is derived using synthesis models as described above. Linking with the above questions surrounding these synthesis models. Then directly leading on from this quote on page 201: `In the optically thick limit it is only necessary to model the bolometric luminosity of the stellar population'
\item

\end{itemize}



%%%%%%%%%%%%%%%%%%%%%%%%%%%%%%%%%%%%%%%%%%%%%%%%%%%%%%%%%%%%%%%%%%%%%%%%%%%%%%%%%%%%%%%%%%%%%%%%%%%%%%%%%%%%%%%%%%%%%%%%



\subsection{Metallicity Measurement}\label{que:metallicity}
\begin{itemize}
\item Line ratios being used to measure metallicity - which ratio is the best? 
\item Accurate abundance measurements require the determination of the electron 
\item What causes the sky background to be so much higher in the NIR than in the optical?
\item The stellar metallicity vs. the gas metallicity. Which is used more as a tracer of galaxy metallicity and which is easier to measure? 
\item What is constituting the continuum emission observed from emission nebulae? Combination of the HI Paschen and Balmer continua and a reflective continuum from starlight scattered by dust.
\item Two dimensional spectral analysis. Beyond the scope of the Maiolino paper.
\item The emission coefficient j
\item The careful avoidance of galaxies hosting an AGN. AGN contribute to the ionising flux of photons and so alter the ratio of emission lines. In this case the metallicity diagnostic diagrams calibrating on star forming galaxies are not usable. How then is the metallicity measured for galaxies hosting AGN? What fraction of galaxies host an AGN? 
\item The lower and the upper branch of the $R_{23}$ ratio - how does the method yield two different metallicity estimates?
\item Quoting the Oxygen abundance in terms of $12 + log\frac{O}{H}$? Why? Where does this come from? 
\end{itemize}

\subsection{Cosmology calculator}\label{que:cosmology_calc}
\begin{itemize}
	\item When computing the angular diameter distance do I also need to code in the cases for a non-flat universe?
	\item What level of precision are we looking for with the different distance measures?
	\item When I use the cosmology calculator with $\Omega _{\Lambda}$ bigger than 1.5 it gives nonsensical answers. Why? 
	\item How to be sure which step size to use when computing these integrals?
	\item Convergence after successive calls to the initial Trapezoidal integral? 
	\item Romberg integration??
	\item Integrating along a given axis?
\end{itemize}

\subsection{Data Analysis}\label{que:data_analysis}
\begin{itemize}
	\item What does simultaneously fitting the emission lines mean? De-blending happens automatically
	\item Where can I find out more about these spectra? i.e. flux units etc 
	\item Why are there so many data units associated with each fits file extracted?
	\item Some of the values show negative flux, how is that possible?
	\item Masking the emission line spikes 
	\item What precision are we wanting here? The task of fitting all of these emission lines is pretty much the subject of an entire SDSS paper and is totally non-trivial
	\item Why are the SFRs described in the galSpecextra table of SDSS reported in terms of PDFs? 
	\item Going through step by step what is the best way to analyse these spectra
	\item Having trouble interpreting the formula in the Kennicutt review for the SFR calibration
	\item SDSS Vacuum corrected wavelengths - only get good fit using these
	\item Difference between compound model which fits the gaussians simultaneously and individual models which fit the gaussians one by one for each of the emission lines? 
	\item Is the error on the flux sigma or is it the FWHM? 
	\item Where should the for loop over the file names be? At the end or within the methods? 
	\item How to compute the equivalent width properly? 
	\item Should the model also include a coefficient to vary the height of the gaussian? i.e. also accounting for absorption?
	\item When the emission spectra are modelled does this also include the emission lines or is it just the continuum emission from the galaxy being synthesised?
	\item When plotting the cross correlation function, how many shifts along the redshift axis should I have? As many as possible? Probably don't want anything more than a second significant figure in the redshift at the moment.	
	\item Are the weak emission lines of high redshift galaxies a result of them being early type, or a result of us being much further away and so the S/N is poorer? 
	\item Cross correlation function - is this for each individual emission line or do I average over all of the flux multiplications?
	\item Why are we doing this SED cross correlation function when we have the galaxy spectrum? Surely it is obvious what the redshift is from the shift of H-alpha?
	\item Constructing the composite spectra?
	\item How to use a weights vector when have error bars on both the x and y axes? 
	\item Starburst99 the two empirical stellar libraries available in the code - one for the milky way and one for the MC. What does this mean? What is a stellar library?
	\item Computing metallicity from the likelihood distribution - Bayesian approach. Specifically in the Tremonti et al. SDSS paper.

\end{itemize}

%%%%%%%%%%%%%%%%%%%%%%%%%%%%%%%%%%%%%%%%%%%%%%%%%%%%%%%%%%%%%%%%%%%%%%%%%%%%%%%%%%%%%%%%%%%%%%%%%%%%%%%%%%%%%%%%%%%%%%%%

\clearpage 
\bibliographystyle{astroads.bst}
\bibliography{/usr/local/texlive/texmf-local/bibtex/bib/ojt.bib}


\end{document}
