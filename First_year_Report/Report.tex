\documentclass{literature}

\title{Exploring the physical properties of high redshift galaxies with KMOS}
\subtitle{First year report}
\author{Owen J. Turner}
\authoremail{turner@roe.ac.uk}
\supervisor{Dr. Michele Cirasuolo}
\supervisoremail{ciras@roe.ac.uk}
\slugheader{First year report}
\slugauthor{Owen J. Turner}
\logo{/Users/owenturner/Documents/PhD/KMOS/Latex/Literature_Review/UoEcrest.pdf}
\graphicspath{{Users/}{owenturner/}{Documents/}{PhD/}{KMOS/}{Latex/}{Literature_Review/}{Figures/}}
\abstract{This first year report is a summary of my progress between September 2014 and May 2015. I have focussed on the development and automation of a bespoke data reduction pipeline for the K-band Multi-Object Spectrograph, and throughout the report the future uses of this pipeline are explained. At present, the pipeline has been automated and tested mainly using stellar data, with the aim to move onto the faint galaxy sample in the near future. The accopanying `Summary' document lists the SUPA courses and teaching responsibilities I have undertaken throughout this time period.}
\setcounter{secnumdepth}{4}


\begin{document}

\coverpage



%%%%%%%%%%%%%%%%%%%%%%%%%%%%%%%%%%%%%%%%%%%%%%%%%%%%%%%%%%%%%%%%%%%%%%%%%%%%%%%%%%%%%%%%%%%%%%%%%%%%%%%%%%%%%%%%%%%%%%


\section{Introduction}\label{sec:Intro}
This report is intended to give an indication of the work I have completed, and the literature relevant to this, between September 2014 and the present date. This work has centred on data obtained using the K-band Multi-Object Spectrograph (KMOS), a near-IR multi-object spectrograph operating on UT4 of ESO's VLT array in Chile. Within this area I have considered two distinct topics. Firstly, I've examined the pipeline through which raw data collected by KMOS are reduced into 1-dimensional science spectra. Secondly, I've focussed on how information is extracted from these spectra and how this information can be related to the physical properties of the imaged objects. Throughout section \ref{sec:background} I give a brief summary of the methods used to measure galactic physical properties, and highlight some of the relationships between these properties which have been observed in populations of galaxies over the last decade. In section \ref{subsec:current_int} a review of the impressive selection of multi-object and integral field spectrographs currently operational is given, before focussing specifically on KMOS throughout section \ref{sec:KMOS}. In section \ref{sec:work} I describe the practical progress made to date, before concluding in section \ref{sec:projects} by discussing the proposed projects involving KMOS data to be undertaken moving into the second year of the PhD. \\    


Currently the main focus of this project is to study the physical processes which govern the formation and evolution of galaxies at $z \sim 3$, a time when the universe was much more active than it is today. It is the gas within these galaxies which is the driving force behind galaxy evolution, and KMOS is designed specifically to measure the dynamical properties and chemical composition of this gas. KMOS is an exciting instrument, with incredible potential to shed light on these questions.   \\ 




\section{Background}\label{sec:background}
Over the last 20 years the way in which we collect and analyse data has changed radically. Telescopes now routinely collect data across the entire spectrum, and ambitious surveys have targeted billions of objects, recording their properties in query-driven relational databases. Statistical techniques have developed to cope with the vast volumes of data output from these surveys, and by applying these we learn about the properties of populations of objects at different times throughout the history of the universe. As the data improves, so do our insights and thus the cycle of technological and computational improvement proceeds. \\

The focus here is on deep galaxy spectroscopy, which is supported by deep narrow-field photometric data. Sections \ref{subsec:deep_photometry} - \ref{subsec:sfr} discuss some of the main methods for recovering physical properties from galaxy spectra and two important correlations observed between these physical properties; the mass-metallicity relation and the fundamental metallicity relation. It is my goal to use these existing methods to recover physical properties using KMOS data, and connections to the planned projects are highlighted wherever relevant. Throughout this report, metallicity will refer to the gas-phase metallicity unless otherwise stated.  


\subsection{Deep Photometric Surveys}\label{subsec:deep_photometry}
In the local universe, wide-field surveys image billions of objects in multiple wavebands across large portions of the entire sky. The current benchmark for such surveys is the SDSS \citep{York2000}, which has imaged $\pi$ sr in five optical wavebands, providing huge amounts of data for a diverse collection of research areas. It is time consuming to collect spectra for large numbers of objects, even in the local universe, and as computational techniques develop an increasing amount can be learnt about the physical properties of an object with photometry alone. With regards to galaxies, this photometry can be used to generate photometric redshift estimates using either training set methods, e.g. Carliles et al. \citep{Carliles2010}, or with SED template fitting e.g. Brammer et al. \citep{Brammer2008}. Ideally, multi-wavelength data from different surveys spanning the UV to x-ray is combined to provide the best possible constraints for the SED fits. Consider Figure \ref{fig:mod_sed}, which shows the model SED for a solar metallicity stellar population at different ages between the UV and far-IR regions of the spectrum. By collecting data which covers this whole range, the metallicity, age, stellar mass and reddening of the stellar population which best fit the grid of models can be selected with greater confidence. \\


\begin{figure}[!htp]
\centering
\includegraphics[width=0.6\textwidth]{Model_SED.png}
\caption{\footnotesize{\emph{The evolution of a modelled solar metallicity stellar population is shown, spanning UV to far-IR wavelengths. Corresponding data is needed to constrain such models and recover the best fitting galactic physical properties \citep{Bruzual2003}}}}
\label{fig:mod_sed}
\end{figure}



In the high redshift universe, it is not possible to conduct such wide surveys, due to objects being significantly fainter and requiring larger exposure times to observe. Instead, focus turns to a small number of `deep fields', which are repeatedly targeted by different surveys so that multi-wavelength data can develop. Largely, the development of these fields is driven by Hubble pointings; in particular the five CANDELS fields \citep{Grogin2011} have become research magnets over the last five years. Each of these fields contains near-IR and far-IR data from Spitzer \citep{Werner2004} and Herschel \citep{Pilbratt2010}, UV and optical data from Hubble, X-ray data from both Chandra \citep{Giacconi2001} and XMM-Newton \citep{Jansen2001} and extensive spectroscopic coverage of the brightest objects from various different surveys. This photometry is ancillary to the spectroscopic surveys, providing the data required to uncover the physical properties which best fit the mdoel SEDs. Figure \ref{fig:deep_surveys} shows a map of selected deep fields from Madau and Dickinson \citep{Madau_2014}, highlighting the physical size subtended at $z\sim 2$ and with CANDELS fields shown in yellow. The KMOS spectroscopic observations for our $z \sim 3$ sample are GOODS-S, COSMOS and UDS fields shown in Figure \ref{fig:deep_surveys}, and also from SSA-22 \citep{Lehmer2009}. \\ 

\begin{figure}[!htp]
\centering
\includegraphics[width=0.6\textwidth]{deep_surveys.png}
\caption{\footnotesize{\emph{The five CANDELS fields are highlighted in yellow, in each case supported by multi-wavelength photometric data and spectroscopy from other surveys. These fields become magnets for research due to the volume and quality of the data \citep{Madau_2014}.}}}
\label{fig:deep_surveys}
\end{figure} 

\subsection{HII Regions and Spectroscopic Surveys}\label{subsec:HII_regions}
Stars emit thermal radiation with a blackbody spectrum that depends on their surface temperature, $T_{*}$. These stars are enshrouded by neutral hydrogen, and any photon with energy greater than the hydrogen ionisation potential, $I_{H} = 13.6eV$, will ionise a hydrogen atom in the region surrounding the star. Take a hot O-type star with surface temperature of $T_{*} = 30,000 K$; the average energy of the thermal photons emitted by the star is $<\bar{E}> \sim k_{B}T_{*} \sim 0.2I_{H}$. This is not enough to ionise hydrogen, but a significant number of photons in the tail of the blackbody distribution will be able to overcome the hydrogen coulomb barrier and liberate an electron. Around such stars, an expanding sphere of ionised hydrogen (HII) develops as each ionising photon encounters a hydrogen atom. Competing with this is the process of radiative recombination, whereby a liberated electron is re-captured to form neutral hydrogen. The re-captured electron cascades through the hydrogen energy levels, producing a smattering of emission line features. Over time the processes of photoionisation and radiative recombination balance one another and the HII region reaches a constant size. Only very hot, very massive and very young stars are capable of producing a significant flux of ionising photons and so nebular emission is a tracer of sights where recent star forming episodes have occured. \\ 



Figure \ref{fig:hii_regions} shows some prominent hydrogen emission lines resulting from a recombination event. Typically H$\alpha$ is the strongest observed line, due to Ly$\alpha$ being subject to significant resonant trapping and subsequent absorption by dust particles. As ISM gas becomes enriched through successive generations of star formation, the nebular regions surrounding hot stars will contain trace amounts of metals. These are also ionised by photons emitted by the stars, resulting in an emission line spectrum which contains information about the chemical enrichment of the galaxy. \\ 
Astronomers use spectroscopy to reveal these emission line features, and hence study the physical properties of galaxies. In the local universe the SDSS \citep{Ahn2014} has produced an unprecedented number of optical galactic spectra, revolutionising our knowledge of the physical processes driving galaxy evolution in the most recent period the history of the universe. Moving to higher redshift, many of these prominent hydrogen and metallic emission lines are shifted into the near-IR, and so to study the same features detectors must be sensitive in this range. Noteably, 3D-HST \citep{Brammer2012} and the Keck MOSFIRE \citep{McLean2012} have conducted large near-IR spectroscopic surveys of high redshift galaxies, and in the future MOONS \citep{Cirasuolo2011} promises to become the high redshift SDSS. Instruments now have high enough resolution to carry out spatially resolved spectroscopy, or are specifically designed as integral-field spectrographs to map the gradients of physical properties across the face of galaxies. The overarching goal of these surveys is to follow the evolution of galactic physical properties over time, linking the physical processes driving galaxy evolution between the epoch of re-ionisation and the local universe.        

\begin{figure}[!htp]
\centering
\includegraphics[width=0.6\textwidth]{hii.png}
\caption{\footnotesize{\emph{ \citep{Ferguson2014}}}}
\label{fig:hii_regions}
\end{figure} 
  
\subsubsection{Dust and Extinction}\label{subsec:dust}
The ISM contains small particles of radii $< 1\mu m$ produced in star forming events, which are revealed both by their ability to extinct stellar/nebular light and by thermal emission in the near-IR. The amount of absorption is measured in magnitudes, $A_{\lambda}$, such that $m_{observed} = m_{true} + A_{\lambda}$. The absorption is wavelength dependent, with bluer light far more susceptable to scattering and absorption from the dust grains. This reddening is quantified by a colour excess, which is the change in colour between two wavebands (often the B and V-bands), due to differential extinction: E($\lambda$ - V) = $A_{\lambda}$ - $A_{V}$. It is incredibly difficult to quantify the amount of dust along a particular sightline, particularly at high redshift, and this can severely affect the measured flux from, and hence the inferred physical properties of observed objects. Commonly, one will assume an empirical extinction curve to correct for both instellar reddening and extinction, the most widely cited of these being the Calzetti curve \citep{Calzetti2000}. These extinction curves are hardwired into stellar population synthesis models, which output an estimate of the true extinction when the models are constrained using multi-wavelength photometry. Despite these empirical and computational tools employed to cope with dust absorption, the whole area is still poorly understood, and the subject of much active research is to reliably estimate the impact of dust on flux observations in both the low and high redshift universe.        	


\subsection{Measuring Stellar Masses}\label{subsec:measuring_mass}
Stellar masses are routinely estimated using stellar population synthesis models, which construct the SEDs of populations of stars. The shape of these SEDs depend on the physical properties of the galaxy, i.e. age, metallicity, redshift and stellar mass. A grid of model SEDs is constructed by varying these physical properties in the population synthesis, and the values are constrained through comparison to an observed SED. The most widely used models are those of Bruzual and Charlot \citep{Bruzual2003} (BC03). \\      
Typically, spectroscopic surveys will target well studied fields to make use of ancillary photometric data. These data preferably cover as wide a region of the spectrum as possible, to provide the information required to break degeneracies between physical properties, such as age and metallicity. As a concrete example, consider the procedure described in Steidel et al. (2014) \citep{Steidel2014}, where stellar masses are estimated for a sample of $z \sim 2$ galaxies. The galaxies have been observed in three optical filters (\textit{$U_{n}GR$}), the near-IR  $ J $ and $ K_{s}$ bands, the HST WFC3-IR F160W filter and the Spitzer/IRAC 3.6$\mu m$ and 4.5$\mu m$ beams. These data cover both the region of the spectrum where rest-frame starlight is emitted directly, and where starlight absorbed by dust is re-radiated. The BC03 models are used to fit the observed SED and stellar masses are inferred by assuming a Chabrier IMF \citep{Chabrier2003}, with estimated uncertainties in log($M_{*} /M_{\odot}$) of $\pm 0.1-0.2 $ dex.    

\subsection{Measuring Metallicity}\label{subsec:abundance}
The chemical composition of the gas fraction of a galaxy is a powerful tracer of its evolutionary history. This gas is the fuel for star formation, wherein heavy elements are cooked via nucleosynthesis and released back into the ISM by stellar winds and supernovae. Thus the abundance of heavy elements in the ISM is linked to the amount of star formation which has taken place within a galaxy. It is necessary to accruately trace the presence of these heavy elements to provide useful constraints for models of chemical evolution. As is common throughout the literature, the oxygen abundance is used as a proxy for the overall gas phase metallicity, and the two terms and used here interchangeably. Giant HII regions were the first indicators of the gas phase abundance of galaxies, and of abundance gradients across the face of these galaxies \citep{Searle1971}, \citep{Shields1974}. Many other abundance indicators have been used since then, including supernova remnants, stellar clusters and AGB stars. HII regions are relatively easy to observe, and the interpretation of their spectra is straightforward, and so these remain the favoured tool for abundance determination. The central question when analysing the spectra from these gaseous regions is always `What is the relationship between nebular line strength and metallicity' and there have been many attempts to calibrate the relationship between emission line ratios and the oxygen abundance. \\

There are two branches of calibration methods, the first being `direct' or 	`T$_{e}$' measurements. The electron density, $n_{e}$, and temperature, $T_{e}$, can be determined by looking at the relative intensities of emission lines produced from different energy levels within the same ion. This works due to the different energy levels depending strongly on $T_{e}$, and the relative populations of the energy levels depending strongly on $n_{e}$. Once these two quantities are known, the abundance ratio between two ions can be determined from the relative intensity between the measured lines, for example: 

\begin{equation}
 	\label{eq:ox_abundance}
 	\frac{O^{++}}{H^{+}} = \frac{[OIII]\lambda5007/H\beta}{j_{[OIII]}(T_{e},n)/j_{H\beta}(T_{e})}
 \end{equation} 

where $j_{[OIII](T_{e}, n)}$ is the emission coefficient \citep{Stasinska2002a}. However, this method of measuring abundances is observationally challenging, with the auroral oxygen line often being undetected or extremely weak, making accurate temperature methods unavailable. This is particularly true at high redshifts and low metallicities, where the HII regions are fainter and the required forbidden lines are redshifted into regions of the spectrum plagued by much higher terrestial background. \\
As a result, the second branch of `strong line' calibration methods developed, relying on empirical relationships between emission line ratios and the metallicites measured either using the $T_{e}$ method or theoretical metallicities computed using photoionisation models \citep{Kewley2002}. The first proposed example of the strong line method comes from Pagel et al. \citep{Pagel1979}, which provides a calibration between the `$O_{23}$' $ = ([OII]\lambda3727 + [OIII]\lambda5007)/H\beta$ ratio and the oxygen abundance. Since then a suite of strong line methods has developed, (e.g. \citep{Pilyugin2001}, \citep{Pettini_2004}, \citep{Pilyugin2005}, \citep{Zaritsky1994}, \citep{Kobulnicky2004}), with each of these based upon different ratios of oxygen, neon and hydrogen emission lines and applying within varying metallicity limits. Part of the reason for this plethora of calibration methods is the redshifting of optical emission features outside of the observable window at ealier epochs, making it unfeasible to apply a single method over cosmic time. Thus to compare results from the local universe with high redshift galaxies, one needs to be confident that the different calibration methods produce metallicities which are in agreement with one another. Liang et al. \citep{Liang_2006} compared the metallicity estimates for $\sim$ 40,000 SDSS galaxies from four different calibrations, finding discrepancies of up to 0.6 dex between empirical methods and those which rely upon photoionisation models. Motivated by the need for an in-depth study of these discrepancies, Kewley et al. \citep{Kewley_2008} compared the metallicity estimates from 10 different calibration methods, finding the same systematic differences reported by Liang et al. Kewley et al. found that is is possible to remove these discrepancies if each of the calibration methods is converted to a single `base' metallicity, by fitting a relationship between each of the methods and the chosen base in turn. The best oxygen abundance is then the average of all of the single base metallicities. By following such a scheme it is possible to compare the results obtained in the local universe using different metallicity indicators. A further issue however is whether this re-normalisation of strong line techniques should be applied to samples of high redshift galaxies\citep{Steidel2014}. Clearly this is a desirable property, but for this to be allowed, the physics of high redshift HII regions would have to resemble the physics of local star-forming galaxies. If there are any significant physical differences, blind application of local calibrations will introduce systematics in inferred metallicity. Perhaps the most challenging situation is the comparison of metallicities derived with one set of strong line indicators in a particular redshift range with those based on a different set of lines at a second redshift. It is a focus of ongoing research to verify the applicability of the above techniques to the high redshift universe, by demonstrating whether the ionisation conditions of high redshift galaxies differ significantly from those locally. \\

The recent advent of integral-field spectrographs has paved the way for measuring spatially resolved metallicities. Recently, Troncoso et al. \citep{Troncoso_2014} have constructed metallicity maps for a sample of 11 galaxies with mean redshift $z \sim 3.4$, four of which are shown in Figure \ref{fig:met_gradients}. One of the main goals using KMOS data will be to construct similar metallicity maps of the observed galaxies, taking into consideration which metallicity calibrations are applicable and whether it is possible to verify any correlation between metallcity and other physical properties such as gas velocity and SFR.   
 

\subsection{BPT Diagram}\label{subsec:BPT-diagram}
The `Baldwin, Phillips, Terlevich' (BPT) diagram \citep{Baldwin1981} is a plane defined by the strong emission line ratios [OIII]/H $\beta$ and [NII] / $H\alpha$, used for classifying the emission line spectra of extragalactic objects. Perhaps the most remarkable aspect of the BPT diagram is the tight locus along which most local star-forming galaxies are found, sometimes referred to as the `HII region abundance sequence'. Tremonti et al. constructed the BPT diagram for a sample of $> 50,000$ local SDSS star-forming galaxies, shown as the grey points in Figure \ref{fig:steidel_bpt}, revealing two branches, the first of which is the locus of star-forming galaxies and the second is the subset of galaxies with AGN activity. Galaxies hosting AGN have much harder UV spectra, the effect of which is to boost the ratio OIII/H $\beta$ relative to NII/H $\alpha$. The BPT diagram is therefore a useful diagnostic test for the presence of these galaxies, which are excluded from abundance analyses due to the contribution of AGN to the values of strong line ratios. \\ 

\begin{figure}[!htp]
\centering
\includegraphics[width=0.6\textwidth]{steidel_bpt.png}
\caption{\footnotesize{\emph{The `BPT' diagram shows the position of star-forming galaxies in the [OIII]/H $\beta$ vs. [NII] / $H\alpha$ plane. This particular figure, taken from Steidel et al. 2014 \citep{Steidel2014}, shows the locus of $z \sim 2.3$ KBSS MOSFIRE star forming galaxies (green) relative to the SDSS $z \sim 0$ galaxies (grey) \citep{Tremonti2004}. The BPT diagram is also used as an AGN diagnostic, with galaxies hosting active AGN being identified as having unusually large [OIII]/H $\beta$ relative to [NII] / $H\alpha$.}}}
\label{fig:steidel_bpt}
\end{figure} 

A crucial goal of high redshift galaxy abudance measurement is to understand the physical origins of the difference in position between the $z \sim 0$ and $z > 1$ locus of star forming galaxies in the BPT diagram. Figure \ref{fig:steidel_bpt} shows the BPT diagram for a sample of local SDSS galaxies \citep{Tremonti2004} and for a sample of galaxies from KBSS-MOSFIRE with $z \sim 2.3$ \citep{Steidel2014}. Clearly there is a vertical offset between the two populations, and Kewley et al. \citep{Kewley2013} argue that the origin of this offset is that high redshift galaxies are bathing in a harder radiation field with higher ionisation parameter. These two effects combine to suggest a weaker dependence of the observed strong line ratios on the chemical abundance of the galaxy. Kewley et al. also conclude that it is not appropriate to use the local AGN classification curve above $z \sim 1.5$, implying that the transition between local and high redshift radiation field conditions occurs somewhere in the region $0.8 < z < 1.5$. Following on from the above section, use of the BPT diagram to separate AGN hosts from a galaxy sample is essential for obtaining accurate metallicity estimates and for diagnosing the redshift ranges between which the physics of the ISM are different to the local norm.    

\subsection{Mass-Metallicity relationship and evolution}\label{subsec:MZ-relation}
In the 1970s a relationship between magnitude and metallicity was discovered \citep{Lequeux1979}, in which brighter galaxies are observed to be more metal rich. In the early 2000s this relationship was undestood to be a manifestation of a more fundamental connection between stellar mass, $M_{*}$, and metallicty, in which galaxies with larger stellar mass have higher metallicities \citep{Garnett2002}, \citep{Tremonti2004}. The origin of this mass-metallicity, (M-Z), relationship is debated, and was investigated locally by Tremonti et al. with a sample of over 50,000 SDSS star-forming galaxies galaxies. The main conclusion of this paper is that the most natural explanation for the positive correlation betwen $M_{*}$ and metallicity is the outflow of enriched material from the galaxy to the ISM. Galaxies with larger mass trap this metal-laden gas in a deeper potential well, and so there is a higher loss fraction in galaxies with lower $M_{*}$. On top of this is the `downsizing' theory, in which there is a systematic dependence of the efficiency of star formation with galaxy mass \citep{Brooks2007}. The combination of these two effects manifest in the local M-Z curve, which is plotted in Figure \ref{fig:steidel_mzr} (a) as taken from the Tremonti paper, which plots the $M_{*}$-metallicity plane for the subset of the SDSS sample showing no indication of AGN activity. The main feature of the best fitting curve to the data is an initially steep rise in oxygen abundance, followed by a flattening above a characteristic $M_{*}$, above which the fraction of enriched gas flowing outwards from the galaxy adopts a shallower dependence upon $M_{*}$. Tremonti et al. continue their analysis to find the gas fraction in bins of $M_{*}$ from their sample. The gas fraction is defined as: 

\begin{equation}\label{eq:gas_fraction}
	\mu = \frac{M_{gas}}{(M_{gas} + M_{*})}
\end{equation}

which requires knowledge of both the stellar and gas mass to compute. As mentioned previously, $M_{*}$ is determined through SED modelling. The gas mass is determined by inverting the Kennicutt-Schmidt law \citep{Kennicutt1998}, which is an empirical relation between then gas density and the SFR in a given region. The result of computing $\mu$ across the different mass bins is the finding that lower mass galaxies tend to have higher gas fractions, a finding which provides a further explanation for the origin of the M-Z relation; it could be a result of low-mass, higher gas fraction galaxies appearing less enriched.  
 \\ 
\begin{figure}[!htp]
\centering
\subfloat[]{\includegraphics[width=0.48\textwidth]{tremonti_mzr.png}}
\subfloat[]{\includegraphics[width=0.48\textwidth]{Maiolino_mzr.png}}
\caption{\footnotesize{\emph{The left hand panel shows the local M-Z relationship, plotted using a sample of over 50,000 local SDSS galaxies, with probably AGN hosts removed using the BPT diagram \citep{Tremonti2004}. The right hand panel shows constructions of the M-Z curve at successively higher redshifts, demonstrating constant shape and a drop in metallicity at fixed $M_{*}$ for higher redshift objects \citep{Maiolino2008}.}}}
\label{fig:steidel_mzr}
\end{figure}


The natural question is to ask whether the M-Z relation evolves with redshift, and hence whether there is any evolution in the way galaxies are processing their enriched gas throughout cosmic time. There are many studies, \citep{Savaglio2005}, \citep{Erb_2006}, \citep{Maiolino2008}, \citep{Henry2013}, \citep{Wuyts_2014}, \citep{Steidel2014} which scrutinise the redshift evolution of the M-Z relation out to $z > 3$, where abundance determinations become much more challenging and less reliable, due to the relationship between emission lines ratios and oxygen abundance being unclear in these differing physical environments. Large uncertainties remain when comparing the M-Z relation recovered by these surveys at different redshifts, due to the use of different metallicity indicators and due to the studied galaxy samples not necessarily forming an evolutionary sequence. Nonetheless, all of the authors recover the trend to observe higher metallicity at higher $M_{*}$, with the curve being systematically offset to lower metallicity values at redshift increases. This is demonstrated in Figure \ref{fig:steidel_mzr} (b), which shows that the shape of the M-Z relation remains constant with redshift, but at a fixed $M_{*}$ the observed metallicity is shifted to lower values. There is debate as to whether this evolution is real, and there is good reason to believe this is the case, with the higher gas fractions of higher redshift galaxies suggesting that there has simply been less time for star formation to produce heavier elements. The counter-argument suggests that the evolution is largely apparent, and a result of a more fundamental 3-dimensional relationship between $M_{*}$, metallicity and SFR. This is the topic of the following section.



\subsection{Fundamental metallicity relationship and evolution}\label{subsec:fmr}
Mannucci et al. \citep{Mannucci2010} have shown that the M-Z relationship is due to a more general relationship between stellar mass, metallicity and SFR, dubbed the Fundamental Metallicity Relationship (FMR). Galaxies within the redshift range $0 < z < 2.5$ define a tight surface within the 3D space defined by these quantities, with a small residual dispersion in metallicity. The existence of this relationship can be explained by the interplay of infall of pristine gas and outflow of enriched material. The former effect is responsible for the dependence of metallicity with SFR and the latter for the mass dependence. Troncoso et al. \citep{Troncoso_2014} argue that since the metallicity of galaxies also depends on their SFR, evolution of the M-Z relation is largely apparent and mostly due to the higher SFRs of galaxies observed at high redshift. This is due both to the preferential selection of brighter, active galaxies at high redshift and to the real intrinsic evolution of the cosmic SFR with redshift. As a result it is more meaningful to discuss the metallicity evolution of galaxies relative to the 3D relation that involves stellar mass as well as SFR. Figure \ref{fig:fmr_new} shows a projection of this relationship, clearly defining a plane upon which all of the sample galaxies spanning $0 < z < 2.5$ can be placed. The introduction of the FMR results in a significant reduction of residual metallicity scatter with respect to the simple M-Z relation. \\ 

\begin{figure}[!htp]
\centering
\includegraphics[width=0.6\textwidth]{fmr.png}
\caption{\footnotesize{\emph{A projection of the FMR space spanned by SFR, metallicity and $M_{*}$. The points without error bars are the median values of metallicity of local galaxies in bins of $M_{*}$ and SFR, and colour coded with SFR according to the colour bar. The squares show high redshift galaxies, demonstrating lower metallicities, higher SFRs and location within the same fundamental plane defined by the low redshift SDSS galaxies.}}}
\label{fig:fmr_new}
\end{figure} 

When taking into account the uncertainties, data up to $z \sim 2.5$ are consistent both in shape and normalisation with the same FMR defined by the local SDSS galaxies, with no evidence for evolution. This lack of evolution spanning 80\% of the lifetime of the universe suggests the FMR is more fundamental than either the M-Z or M-SFR relations, and that it is directly related to the physical mechanisms governing galaxy formation. Above $z \sim 2.5$ galaxies appear to evolve away from the FMR to lower metallicity values. A plethora of potential observational biases are listed in Mannucci section 5.1 \citep{Mannucci2010}, but barring these being completely responsible for the metallicity shift, the divergence from the FMR suggests a transition to a period were different physical processes dominate galaxy evolution, i.e. smooth accretion of cold gas is no longer the main regulatory mechanism. \\ 
It will be possible to measure SFRs, metallicities and $M_{*}$s using KMOS data at $z \sim 3$ and one of our goals is to look at how deviations from the FMR correlate with the dynamical state and morphology in this redshift range.   


\subsection{Measuring Star Formation Rates}\label{subsec:sfr}
Throughout their lifetimes, galaxies convert bound gas into stars, and do so at a varying rate. The star formation rate is an indicator of the efficiency with which stellar mass is being built up within a galaxy, and is dependent upon galaxy type, environment and cosmic epoch. Observations have revealed that most stars form within galaxies that follow a relatively tight SFR-M$_{*}$ `Main Sequence'. The existence of this sequence suggests that the evolution of the global Star Formation History (SFH) of the universe is determined by a balance between gas accretion and feedback to the IGM, such as stellar winds, supernovae shocks and AGN activity. Understanding these processes which regulate the gas volume and composition within a galaxy, and seemingly drive the changes in star formation rates over time is a fundamental aspect of galaxy evolution. \\ 
Here an overview of the observational methods employed to measure the SFRs of galaxies is given. 
Sections \ref{subsubsec:uv_continuum} - \ref{subsubsec:comp} describe the individual SFR tracers, the motivation behind their use and the benefits and downfalls of applying each tracer independently. Underpinning all of these techniques are Stellar Population Synthesis (SPS) codes, which use the equations of stellar evolution to model the light emitted by a population of stars at different ages. The spectrum of a galaxy is determined by the ratio of early-type to late-type stars, and so the observed colour of a galaxy contains information about the star formation over the most recent stage of the galaxy's history. By modelling galaxy colours as a function of SFR, age, composition and IMF, SPS codes provide a calibration between the observed luminosity of a particular SFR tracer and the SFR of the galaxy. Section \ref{subsubsec:gsfh} will summarise the now consistent picture of the global SFH of the universe, highlighting the peak period of star formation at $z \sim 2$.  \\ 

The advent of NIR integral-field spectrographs on 8-10m class telescopes has paved the way for studying spatially resolved SFRs within galaxies during the time when the universe was at its most active. Spatially resolved spectroscopy is crucial for uncovering the dependency of SFR upon environment, rather than making an assumption that the galaxy properties are similar across its spatial extent and deriving a holistic SFR. Section \ref{subsubsec:spat_res} will discuss this in more detail, focussing on the opportunities presented by these instruments and the observational challenges which must be overcome to reach these goals.     


\subsubsection{The UV continuum}\label{subsubsec:uv_continuum}
The wavelength range 1200-2500\AA is longward of the Lyman-continuum break and bluewards of the contaminating flux emitted by older stellar populations. The near-UV continuum emission of galaxies in this range traces the photospheric emission of young stars and is thus one of the most direct methods of measuring galaxy SFRs. This wavelength range is shifted into the easily observed optical range up until $z\sim 3-4$, and so can be applied over a wide range of redshifts. As such, UV continuum measurements provide a powerful probe of the cosmic evolution of the SFR. There are two main causes for concern when applying this method. The first is that this wavelength region is particularly sensitive to dust extinction, and it is particularly hard to infer what fraction of the total light is being observed. The second is that the models used to calibrate the UV luminosity - SFR conversion are sensitive to the form of the IMF assumed. \\  

For extragalactic studies, this method has been revolutionised by the launch of Galaxy Evolution Explorer mission (GALEX, \citep{Martin2005}). GALEX imaged two thirds of the sky in near-UV (230nm) and far-UV (155nm) channels, providing integrated measurements of distant galaxies and resolved mapping for our nearest external galaxies. The impact of this is the recovery of UV flux measurements for hundreds of thousands of galaxies, which can be used to calibrate the relationship between luminosity and SFR and also to provide fresh data to constrain the relationship between the UV spectral slope and dust attenuation.      

\subsubsection{Infrared Emission}\label{subsubsec:ifrared}
A significant fraction of the bolometric luminosity of a galaxy is absorbed by dust and re-radiated in the thermal IR, at wavelengths between 10-300$\mu m$. The dust absorption cross section is strongly peaked in the UV, and so in principle the FIR emission can be a sensitive tracer of the young stellar population and SFR. However, there is a varying contribution to the UV light from the young stellar population in different types of galaxy, affecting the efficacy of the FIR luminosity as a SFR tracer. For example, the FIR luminosity is the ultimate tracer of the SFR in starburst galaxies, where young stars completely dominate the UV continuum and the dust opacity is high everywhere. In this situation there can be no hope of applying individually an UV tracer of the SFR, as the light is mostly absorbed. Alternatively, consider a normal galaxy with both a warm and cold dust component. Here, heating of the dust from the old stellar population, even in the rest-frame UV, is a significant effect and confuses the relationship between IR-luminosity and SFR. \\ 

The Spitzer Space Telescope \citep{Werner2004} and the Herschel Space Observatory \citep{Pilbratt2010} have been transformational in developing a better understanding of the connection between IR-continuum emission and SFR. The crucial point is that the conversion between IR-luminosity and SFR must change when observing at different IR wavelengths and when observing different galaxy classes.   

\subsubsection{Emission Line Tracers}\label{subsubsec:em_lines}
Particularly relevant to this work are SFR tracers derived from nebular emission lines. Clouds of hydrogen surrounding a star forming region are ionised by energetic photons emitted by young and massive stars. The recombination of an electron back into the hydrogen atom is coupled with the emission of a photon. In this sense, nebular emission lines are effectively re-emission of the integrated stellar luminosity shortwards of the Lyman limit, so they provide a direct sensitive probe of the young stellar population. Only stars with $M > 10M_{\odot}$ ($\tau _{MS} < 10Myr$) contribute significantly to the ionising photon flux and as a result emission line tracers are particularly sensitive to the instantaneous integrated SFR of a galaxy. The strongest and therefore easiest to observe recombination lines are in the rest-frame optical, and so application of this technique to high redshift requires near-IR surveys. Special importance is placed on the H$\alpha$ line as a tracer - it is almost always the strongest emission line feature in a galaxy, and so the conversion between $H\alpha$ luminosity and SFR is well studied. Again, dust extinction and the form of the IMF will affect SFRs recovered and implicit in this technique is the assumption that all massive star formation is traced by the ionised gas, which may not necessarily be the case. \\ 

H$\alpha$ is redshifted out of the optical range beyond $z\sim 0.5$, and so there is significant motivation to calibrate weaker lines in the hydrogen series and metal emission lines bluewards of this transition. However, these lines are orders of magnitude weaker than H$\alpha$ and are difficult to observe at high redshift. This is an area which will see leaps forward with the advent of JWST, especially in the observation of weaker, but less extinguished, higher order hydrogen recombination lines. There is a case for using Lyman-$\alpha$ as a SFR tracer, in principle very appealing as this can be applied at high redshift and the line strength is roughly 9 times that of H$\alpha$. However, the line is subject to strong quenching from resonant trapping and eventual absorption by dust, usually quantified by a Lyman-$\alpha$ escape fraction ($f_{esc}$). Without detailed knowledge of $f_{esc}$, Lyman-$\alpha$ is an ineffective tracer, and \citep{Hayes2011} have shown that $f_{esc}$ appears to vary unpredictably.  


\subsubsection{Radio and X-ray Tracers}\label{subsubsec:rad_x}
There have also been attempts to calibrate relationships between X-ray and radio luminosities and SFRs. AGN emit strongly in both of these regimes, however if the stellar component can be isolated, in both cases it is found to correlate with the IR emission. This correlation can be used to bootstrap conversions between the radio and x-ray luminosities of a galaxy and SFR.

\subsubsection{Composite Multiwavelength Tracers}\label{subsubsec:comp}
Clearly there are some situations where one independent tracer of the SFR is better than another, or where different tracers are complimentary. Composite tracers are calibrated relations that combine information from more than one different wavelength band to arrive at a SFR estimate. This has recently become a much more feasible approach, with large multiwavelength surveys collecting the data required to calibrate these types of relations using SPS models. The most widely explored composite tracer uses a linear combination of the UV and IR continuum luminosity to arrive at a `corrected' UV luminosity, $L_{UV}(corr)$. This can then be used in conjunction with the usual calibration to arrive at a SFR. Typically, composite tracers dramatically reduce the errors that would be found using a single SFR tracer independently.   


In sum, Kennicutt and Evans \citep{Kennicutt_2012} tabulate the conversion between luminosity and SFR for each of the tracers listed above. Figure \ref{fig:comp_convert} lists the units of the observed luminosities, $L_{x}$, in the different wavebands, and the value of the constant, $C_{x}$, used to convert to SFR. This conversion is given by Equation 3 below. The composite tracers are also listed in Figure \ref{fig:comp_convert}, providing a list of prescriptions to find the composite `corrected' luminosities. 

\begin{equation} \label{eq:convert}	
	log\dot{M_{*}}(M_{\odot}year^{-1}) = logL_{x} - logC_{x}
\end{equation}



\begin{figure}[!htp]
\centering
\includegraphics[width=0.8\textwidth]{convert.png}
\includegraphics[width=0.8\textwidth]{comp_tracers.png}
\caption{\footnotesize{\emph{The top panel lists conversion factors between observed luminosity and SFR for different tracers, and the bottom panel lists the calibrated SFR tracers using combined multi-wavelength data}}}
\label{fig:comp_convert}
\end{figure} 


\subsubsection{The Global Star Formation History}\label{subsubsec:gsfh}
Overall, having a handle on the global SFR of the universe at different epochs tells us about the activity of the universe at these times. We appear to live at a time which is much less active than in the past, with stars forming at rate 9 times lower than observed at the peak of the SFRd curve \citep{Madau_2014}. The obvious question that needs to be answered is why this is the case. Have galaxies rifled through their gas budgets and produced the vast majority of their stellar mass? Are we now just coasting along in a relatively dormant universe until the current generation of main sequence stars run out of fuel and die? \\ 

These statements are drawn from a knowledge of the SFH of the universe, a current computation of which is given in Section 5 of Madau and Dickinson 2014 \citep{Madau_2014}. The authors use a sequence of determinations of the UV and IR luminosity functions at increasing redshift, integrating over these to find the comoving luminosity density at each redshift. These can then be converted into SFRDs at each redshift using the factors given in Figure \ref{fig:comp_convert}. Figure \ref{fig:sfh} plots these measurements of the SFRD in the UV and IR respectively, plots a) and b) respectively, before plotting both on the same axes in plot c). All three plots are accompanied by a best fitting function. The key feature of this well known diagram is the rise to peak SFRD between $8 < z < 2$, and subsequent decline up until the present. Structure is forming during the initial rise to peak SFRD, and massive galaxies efficiently churn through their gas budgets in the `downsizing' scenario \citep{Brooks2007}. Beyond the peak, the gas is essentially used up and galaxy SFRs decline steadily until the current epoch. \\ 

\begin{figure}[!htp]
\centering
\includegraphics[width=0.8\textwidth]{sfh.png}
\caption{\footnotesize{\emph{The history of cosmic star formation from a) FUV b) IR and c) FUV + IR. The density of star formation rises until peak at $z\sim 2$, the time when the universe was most active, and has been dropping off since then.}}}
\label{fig:sfh}
\end{figure} 

A remarkably consistent picture of the SFH of the universe has emerged, which has given an overall perspective of the activity of the universe as galaxies have evolved. There is still work to be done constraining the shape of this curve, particularly at the high redshift end where there are few data points. Adding in the integrated luminosity functions derived from the highest redshift galaxies will help cement the SFRD curve during the epoch of galaxy formation. 

\subsubsection{Spatially Resolved SFRs}\label{subsubsec:spat_res}
Nearly all the diagnostic methods described up to now have been designed for measuring integrated SFRs of galaxies, implicitly assuming that the local variations in stellar-age, mix, IMF population and gas/dust geometry largely average out when the integrated emission of a galaxy is measured. As mentioned above, with very high resolution imaging, or integral-field spectroscopy, (see section \ref{subsec:current_int}), one would like to create maps of spatially resolved star formation by considering individual patches of the whole galaxy. The problem is that at small linear scales almost all of the statistical approximations embedded into SPS models begin to break down. Firstly there is the problem of stochasticity, where incomplete sampling of the stellar IMF on scales between 0.1-1kpc lead to large variations in tracer luminosity. A good example of this is given in section 3.9 of Kennicutt $\&$ Evans \citep{Kennicutt_2012}, in the context of what a theoretical observer in M51 would infer from the outside when looking at different regions of the Milky Way. Secondly, if the spatial resolution of the SFR measurements encompass single young clusters, the assumption of continuous star formation embedded in the SFR recipes breaks down. The emission from these regions is completely dominated by a very young stellar population, with ages less than the averaging times assumed for all but the emission line tracers. Again, the consequence is that the luminosities of the emission line tracers are affected. Thirdly, if the resolution of measurements becomes fine enough to enter within the Stromgren diameter of HII regions, indirect tracers tend to map out cloud emission rather than the young stars. \\ 

\begin{figure}[!htp]
\centering
\includegraphics[width=0.6\textwidth]{tron_gradients.png}
\caption{\footnotesize{\emph{[OIII]5007, [OII]3727, H$\beta$ flux maps and the [OIII]5007/[OIII]3727 ratio as well as metallicity maps for four galaxies from the AMAZE sample \citep{Troncoso_2014}. In each case the peak abundance is anti-correlated with the peak of star formation, suggesting that infall of pristine gas towards the centre of the galaxy is both diluting the gas and fueling star formation.}}}
\label{fig:met_gradients}
\end{figure} 

These three biases complicate spatially resolved star formation rate measurements, but do not rule these out. One of the goals of this project will be to construct spatially resolved maps of the star formation within galaxies, inspecting how this correlates with the metal content in the same regions. Previous studies have used integral-field spectrographs to construct 2D maps of various emission line features, hence allowing for an estimation of the SFR and oxygen abundance in the same regions. Notably, Cresci et al. \citep{Cresci2010} published maps of the spatially resolved [OIII]5007 intensity, velocity profile and oxygen abundance within a subset of 3 galaxies from the AMAZE sample \citep{Maiolino2008} of $z \sim 3.4$ galaxies. Expanding upon this, Troncoso et al. \citep{Troncoso_2014} plotted similar maps for a larger collection of AMAZE galaxies, and also included the spatial distribution of SFR in their analysis. Figure \ref{fig:met_gradients} shows these maps for four galaxies in the Troncoso paper. Interestingly both Cresci and Troncoso report an anti-correlation between oxygen abundance and SFR, with SFR typically declining outwards from centre and oxygen abundance showing the opposite. This is suggestive of massive inflows of pristine gas into the core of the galaxy, which both fuels star formation and dilutes the metal content in these regions. A larger sample is required to reveal whether this phenomenon is ubiquitous at high redshift, and whether the dynamical state of the galaxies or mergers have any role to play in observed abundance gradients. Creating an analogous set of maps for a larger sample of $z \sim 3$ galaxies using KMOS data will be one of key projects of my second year, and this topic is discussed in section \ref{subsec:redshift_three_sample}.


\subsection{Current multi-object and integral field spectrographs}\label{subsec:current_int}
	
Emphasis is now being placed on the role Multi-Object Spectrographs (MOSs) play in revealing the properties of object populations. For years, single object spectrographs have revealed a huge amount about the properties of small numbers of these objects. However, detailed knowledge of the behaviours of object populations is difficult to grasp without the statistical power of larger collections of spectra, which sample different object environments and reveal a wider spread in galaxy physical properties. Particularly for ground based observations in parts of the spectrum strongly affected by atmospheric effects, some of the most exciting applications of spectroscopy involve very long integration times, even on 10m telescopes. By simultaneously collecting spectra for many objects, MOSs promise to collect much larger samples of spectra from which more robust conclusions can be drawn. \\

Integral Field Spectrographs (IFSs), or spectrographs equipped with Integral Field Units (IFUs) combine the capacity for imaging and for spectroscopy. The result is resolved spectroscopy, in which a galaxy image is split into a grid, and each segment of the grid is dispersed using a grating to obtain a spectrum. This collection of spectra for a single galaxy image is called a datacube, which contains information about how the physical properties of the galaxy change across its radial extent. Studies involving resolved spectroscopy are crucial for understanding the gas, and by extension galaxy, evolution, through looking at abundance and SFR gradients and by directly examining the gas dynamics. The current observational benchmark is the combination of both integral field and multi-object spectroscopy with a single instrument, providing large samples of resolved spectra. This is the optimal research tool for developing an understanding of the physical processes driving galaxy formation and evolution. At increasingly high redshifts, the rest-frame optical emission lines used to measure abundance and trace gas velocity are shifted to longer wavelengths. As a result, to develop a coherent view of galaxy evolution throughout the cosmic epochs it is necessary to have instruments sensitive across the optical and near-IR regions of the spectrum.  \\ 

Sections \ref{subsubsec:MUSE} - \ref{subsubsec:MOONS} highlight spectrographs which are currently operational with either integral field or multi-object capabilities, indicating the passbands in which they are sensitive.In section \ref{sec:KMOS} focus turns to the K-band Multi-Object Spectrograph (KMOS), which is the instrument used throughout this work.     

\subsubsection{MUSE}\label{subsubsec:MUSE}
The Multi-Unit Spectroscopic Explorer (MUSE) \citep{Bacon2010} is an optical mutli-object integral field spectrograph, with 24 IFUs, operating at the Nasmyth focus of Unit Telescope 4 (UT4), part of ESO's Very Large Telescope (VLT) array in Chile. MUSE covers the spectral range 0.45-0.95$\mu m$ and will produce over 90,000 spectra with each observation, making it an extremely powerful instrument for resolved spectroscopy. The science objectives of MUSE range between the local and high redshift universe, from in depth studies of star formation, mergers, gas kinematics and supermassive black holes locally to high redshift studies of Lyman$\alpha$ emitters, re-ionisation and the cosmic web.

\subsubsection{MOSFIRE}\label{subsubsec:MOSFIRE}
The Multi-Object Spectrometer For Infra-Red Exploration (MOSFIRE) \citep{McLean2012} is a MOS for the Keck observatory in Hawaii. KBSS-MOSFIRE is a deep near-IR survey covering 15 fields of the Keck Baryonic Structure Survey (KBSS), specifically targeting the redshift range $2.0 < z < 2.6$, a range in which a relatively complete set of the rest-optical nebular lines fall fortuitously within the near-IR atmospheric windows for ground based spectroscopy. As a result, this survey is optimised for collecting spectra for and examining the properties of star forming galaxies during the peak of the SFRd curve. On top of this, the MOSFIRE Deep Evolution field (MOSDEF) \citep{Kriek2014}  aims to collect $\sim 1500$ rest-frame optical spectra in three well studied Cosmic Assembly Near-Infrared Deep Extragalactic Legacy Survey (CANDELS) \citep{Grogin2011} fields.

\subsubsection{SINFONI}\label{subsubsec:SINFONI}
The Spectrograph for INtegral Field Observations in the Near Infrared (SINFONI) \citep{Eisenhauer2003} is an IFS operating at the Cassegrain focus of UT4 at the VLT array. The spectroscopic imaging survey in the near-infrared with SINFONI (SINS) collected the largest sample of spatially resolved gas kinematics, morphologies and physical properties of star forming galaxies at $z \sim 1-3$ \citep{ForsterSchreiber2009}, dubbed the `AMAZE' sample. Recently the spatially resolved properties of the AMAZE sample were published by Troncoso et al. \citep{Troncoso_2014}, which is a feat we seek to better using a larger and statistically move robust sample of $z \sim 3$ KMOS galaxies.

\subsubsection{VIMOS}\label{subsubsec:VIMOS}
The Visible Multi-Object Spectrograph (VIMOS) \citep{LeFevre2003} is an optical wide field imager and MOS, operating between 0.36-1$\mu m$, mounted on the Nasmyth focus of UT3 at the VLT array. The Vimos VLT Deep Survey (VVDS) collected thousands of spectra from the Chandra Deep Field South area, \citep{LeFvre2004}, \citep{LeFevre2005}, which are crucial in their own right, but also act as important calibration data for photometric redshift techniques \citep{Ilbert2006}. The main science goals of VIMOS are to study the formation of galaxies during the early stages of the universe and to shed light on large scale structure and dark energy by collecting thousands of high redshift galactic spectra. 


\subsubsection{SAMI}\label{subsubsec:SAMI}
The Sydney-AAO Multi-object Integral field spectrograph \citep{Croom2011} is an optical multi-object IFS operating on the Sydney Anglo-Australian telescope. SAMI will collect spatially resolved spectroscopy up to $z \sim 0.1 $ for between 10,000 and 100,000 galaxies, in order to study the spatial gradients of physical properties for a statistically significant sample of galaxies in the local universe. 


\subsubsection{LUCIFER}\label{subsubsec:LUCIFER}
The LBT NIR-Spectroscopic Utility with Camera and Integral Field unit for Extragalactic Research \citep{Mandel2000} is a near-IR IFS operating between 0.9-2.5$\mu m$. The instrument is operation on the Large Binocular Telescope (LBT) on Mt. Graham, Arizona, and its main purpose is to study the dynamical properties and chemical gradients in high redshift galaxies. Like KMOS, the specifications of LUCIFER also make it a useful instrument for studying populations of objects in the local universe, for example resolved stellar spectroscopy. 


\subsubsection{GMOS}\label{subsubsec:GMOS}
The Gemini Multi-Object Spectrograph (GMOS) is a multi-object IFS installed on the two 8m Gemini Observatory telescopes in Hawaii and Chile \citep{Davies1997}. The instrument is sensitive in the optical range, between 0.36-0.94$\mu m$ making it ideal for resolved spectroscopy in the local universe. The science objectives of this instrument including studying stellar and gas kninematics, black-hole binaries and chemical abundance gradients across the faces of galaxies. 


\subsubsection{MaNGA}\label{subsubsec:MaNGA}
Mapping Nearby GAlaxies is a new survey at Apache Point Observatory as part of the Sloan Digital Sky Survey IV (SDSS-IV), with the aim of obtaining integral-field spectroscopy for an unprecendented sample of 10000 nearby galaxies \citep{Bundy2014}. MaNGA's key goals are to understand the `life cycle' of present day galaxies from imprinted clues of their birth and assembly, through their ongoing growth via star formation and merging, to their death from quenching at late times. To achieve these goals, MaNGA will channel the impressive capabilities of the SDSS-III BOSS spectrographs in a fundamentally new direction by using the unique power of 2D spectroscopy.

\subsubsection{MOONS}\label{subsubsec:MOONS}
The Multi-Object Optical and Near-IR Spectrograph is a construction stage instrument, which is planned to be operational by 2019. MOONS spans both the optical and near-IR regions of the spectrum, covers a very large field of view and can simultaneously collect 500 spectra \citep{Cirasuolo2011}. These characteristics make MOONS perfectly suited for studying galaxy formation and evolution throughout the entire history of the universe, making it a high redshift spectroscopic equivalent to SDSS. As well as this, the high spectral resolution of the instrument will allow astronomers to study the chemical abundances of stars in our galaxy, particularly in the obscured bulge. 


 
\section{KMOS}\label{sec:KMOS}
As detailed in the above sections there is an expanding selection of impressive instrumentation available both for multi-object and for integral field spectroscopy. It is worthwhile to focus on where KMOS fits in to the big picture. Section \ref{subsec:instrument} provides a description of the instrument design and the crucial science drivers motivating its construction and section \ref{sec:reduction} describes the key recipes of the ESO reduction pipeline for processing KMOS raw images and reconstructing three dimensional data cubes. 

\subsection{Specification, science drivers and context}\label{subsec:instrument}
The details of the physical processes driving galaxy formation and evolution in the first few billion years after the big bang remain elusive. The ideal set of operational capabilities for an instrument seeking to shed light on these topics are as follows. 
\begin{itemize}
	\item The ability to collect spectra for many objects simultaneously; a statistically robust sample will be difficult to assemble with single object IFUs alone 
	\item The ability to obtain more than just integrated or one dimensional information, since forming galaxies are often observed to have complex morphologies 
	\item Sufficient resolving power to detect the relatively small differences in velocity observed in galaxy rotation curves
	\item The flexibility to target objects clustered together in small areas of the total field of view 
	\item The ability to observe well studied optical features at high redshift in the near-IR Y, J, H and K-band atmospheric windows    
\end{itemize}

These are in essence the key features of KMOS, which is a near-IR multi-object integral field spectrograph operating with 24 configurable pickoff arms in the Nasmyth focal plane. These arms position mirrors at user specified locations, which feeds the light through to 24 image slicer IFUs that position each sub-field into 14 identical slices with 14 spatial pixels along each slice. Light from the IFUs is then dispered by three identical cryogenic grating spectrometers to generate 14x14 spectra for each sub-field, with the dispersed light reaching three identical 2kx2k Hawaii 2RG HgCdTe detectors. Each spatial pixel corresponds to 0.2$^{\prime \prime}$, giving each IFU a 2.8$^{\prime \prime}$x2.8$^{\prime \prime}$ sub-field. Each arm patrols the 7.2$^{\prime}$ diameter of the total field of view of the telescope. This system offers significant advantages over multi-slit spectrographs, because of the reduced slit contention in crowded fields, and the sensitivty of the pickoff arms to galaxy morphology and orientation. KMOS operates in 5 near-IR bands, these being the IZ, YJ, H, K and HK, with varying spectral resolution between these \citep{Sharples2005}. In short, this is an instrument perfectly equipped to map variations in star formation histories, merger rates and dynamical masses across a wide range of redshifts and environments, with these properties probing the history of galaxy evolution in the universe. \\ 

Figure \ref{fig:raw_image} shows an example of a raw KMOS image from one of the three detectors. The detector is divided into 8 columns, these being the designated spaces for light from each of the 8 IFUs dedicated to this detector. In this exposure, one of the IFUs in the middle of the detector is not being illuminated, either as a result of not having a target assigned or because of a mechanical fault. The y-direction of the detector is the wavelength axis, spanning 2048 pixels in total. The x-direction pixels contain information about both the x and y-directions on the sky. Each column is divided into 14 slitlets, and each slitlet contains 14 pixels, together forming a 14x14 pixel grid for each target, at each of the 2048 spectral pixels. Clearly visible as vertical stripes in some of the IFUs are the target objects, and the horizontal arcs spanning all IFUs are bright emission features from the OH molecule within our atmosphere. These OH lines are a nuisance, and must be removed by subtracting a sky image from each object image. Due to target/IFU size, it is not possible to simultaneous collect a sky and object image within individual IFUs. The KMOS observing strategy follows an Object-Sky-Object (O-S-O) dithering pattern, where the IFU arm nods to a blank patch of sky between object exposures. The implication is that a single sky image is being used for two object images subtractions, increasing the efficiency of the OB. The sky subtraction process and the extent of the OH line problem is discussed in detailed through section \ref{subsubsec:Oh_lines}.

\begin{figure}[!htp]
\centering
\includegraphics[width=0.8\textwidth]{raw_image.png}
\caption{\footnotesize{\emph{Example raw image from one of the KMOS detectors}}}
\label{fig:raw_image}
\end{figure}  	
  
\subsection{Data Reduction Pipeline}\label{sec:reduction}
By virtue of being a multi-object integral field spectrometer with 24 configurable IFUs and over 4000 spectra per exposure, the raw data taken by KMOS, and hence the reduction pipeline, are complex. Fortunately, ESO has made available the Software Package for Astronomial Reductions with KMOS (SPARK) \citep{Davies2013}, for carrying out the bulk of this reduction. The pipeline scripts and documentation are available to download from the ESO pipeline installation pages \citep{ESO2015}. Included in this package are `recipes' for processing dark frames and flatfields, for wavelength calibration, illumination correction, flux calibration and for reconstructing and combining data cubes using these calibrations. These recipes are decsribed briefly throughout sections \ref{subsubsec:dark_frames} - \ref{subsubsec:sci_red}, although this is just a summary of the SPARK Instructional guide I relied upon heavily whilst learning the stages of the pipeline. This guide is part of the documentation available from the download page \citep{ESO2015}. The Common Pipeline Library (CPL) is a framework used by ESO for the creation of automated data reduction tasks. The philosophy behind the CPL is to standardise VLT instrument pipelines, shorten their development and ease their maintenance. One of the key features provided by the CPL is the ability to create data reduction algorithms that run as plugins, known as recipes. The ESO Recipe Execution Tool (EsoRex) is a central hub for listing, configuring and executing CPL based recipes from the command line. Whilst using the KMOS pipeline, recipes are executed via EsoRex from within Python scripts, allowing for easy automation of the different processing steps. The recipes described throughout the following sections require calibration data taken by KMOS. At the end of each night, the relevant set of calibration images are taken at six different rotator angles, to allow for the closest match to the rotation angle of the data as possible. These calibration files, as well as the object and sky data, are assigned their type within the fits header of the file, e.g. DARK, FLAT\_OFF, ARC\_ON. Each of the recipes requires a Set Of Files (SOF) as input, which are a combination of raw calibration data, static calibration files provided with the pipeline installation and calibration products from the execution of previous recipes. The recipes are executed from the command line by typing `esorex \* Recipe\_Name \* \* Recipe\_SOF \*', where the Recipe\_SOF is a text file containing the names of the input products for each recipe.   \\ 

\subsubsection{Dark Frames}\label{subsubsec:dark_frames}
\textbf{\underline{Recipe Name}} \\		
\noindent
{\tt{kmos\_dark}} \\

\noindent
\textbf{\underline{Purpose}} \\	
\noindent 
To identify `hot' and `dead' pixels on the detector using a series of dark exposures. These are pixels which are inactive or which 
read unphysically large electron counts due to thermal excitation within the instrument. \\

\noindent
\textbf{\underline{Input Files}} \\	
\noindent 
3 {\tt{DARK}} type raw images of the same exposure length. \\ 

\noindent
\textbf{\underline{Output Files}} \\	
\noindent
{\tt{badpixel\_dark.fits}} - map of bad pixel locations on the three detectors \\

\subsubsection{Flatfields}\label{subsubsec:flatfields}
\textbf{\underline{Recipe Name}} \\		
\noindent
{\tt{kmos\_flat}} \\


\noindent
\textbf{\underline{Purpose}} \\	
\noindent 
To account for pixel-to-pixel sensitivity across an image by exposing each of the detectors to a uniform source of light.\\

\noindent
\textbf{\underline{Input Files}} \\	
\noindent 
3 {\tt{FLAT\_OFF}} exposures with the shutter closed \\ 
\noindent
18 {\tt{FLAT\_ON}} exposures taken at six different rotation angles \\
\noindent 
{\tt{badpixel\_dark.fits}}

\noindent
\textbf{\underline{Output Files}} \\	
\noindent
{\tt{master\_flat\_\#\#\#.fits}} - map of bad pixel locations on the three detectors (\#\#\# represents the filter used for the calibrations)\\
\noindent
{\tt{xcal\_\#\#\#.fits}} - The x coordinates for every illuminated pixel within an IFU\\
\noindent
{\tt{ycal\_\#\#\#.fits}} - The y coordinates for every illuminated pixel within an IFU\\
\noindent
{\tt{badpixel\_flat\_\#\#\#.fits}} - map of bad pixel locations on the three detectors\\
\noindent
{\tt{flat\_edge\_\#\#\#.fits}} - coefficients of the fits to the left and right edges of the slitlets in the IFUs

\subsubsection{Wavelength Calibration}\label{subsubsec:wavelength_calib}
\textbf{\underline{Recipe Name}} \\		
\noindent
{\tt{kmos\_wave\_cal}} \\

\noindent
\textbf{\underline{Purpose}} \\	
\noindent 
To assign a wavelength value to each of the spectral pixels in the chosen waveband using a set of arc frames and static calibration files \\

\noindent
\textbf{\underline{Input Files}} \\	
\noindent 
1 {\tt{ARC\_OFF}} type raw file \\ 
\noindent 
6 {\tt{ARC\_ON}} type arc exposures of the required waveband for the 6 different rotator angles \\ 
\noindent
{\tt{master\_flat\_\#\#\#.fits}}\\
\noindent
{\tt{xcal\_\#\#\#.fits}}\\
\noindent
{\tt{ycal\_\#\#\#.fits}}\\
\noindent
{\tt{badpixel\_flat\_\#\#\#.fits}}\\
\noindent
{\tt{flat\_edge\_\#\#\#.fits}}\\
\noindent 
{\tt{kmos\_ar\_ne\_list\_\#.fits}} - static calibration file, list of neon emission line wavelengths in the required waveband\\
\noindent
{\tt{kmos\_wave\_band.fits}} - static calibration file, filter limits in each of the wavebands \\
\noindent
{\tt{kmos\_wave\_ref\_table.fits}} - static calibration file, reference wavelengths in each of the wavebands\\



\noindent
\textbf{\underline{Output Files}} \\	
\noindent
{\tt{lcal\_\#\#\#.fits}} - frame containing the wavelength in microns for every illuminated pixel on the detector\\
\noindent
{\tt{det\_img\_wave\_\#\#\#.fits}} - reconstructed arc-lamp frame \\

\subsubsection{Illumination Correction}\label{subsubsec:illum_cor}
\textbf{\underline{Recipe Name}} \\		
\noindent
{\tt{kmos\_illumination}} \\

\noindent
\textbf{\underline{Purpose}} \\	
\noindent 
To check the uniformity of the field of view illumination during the flatfield exposures. Note that this is not a mandatory step, and all following recipes will run if they are not supplied with the output product from this recipe.\\

\noindent
\textbf{\underline{Input Files}} \\	
\noindent
3 {\tt{FLAT\_SKY}} type exposures of the required waveband \\ 
\noindent
{\tt{master\_dark\_\#\#\#.fits}}\\
\noindent
{\tt{master\_flat\_\#\#\#.fits}}\\
\noindent
{\tt{xcal\_\#\#\#.fits}}\\
\noindent
{\tt{ycal\_\#\#\#.fits}}\\
\noindent
{\tt{lcal\_\#\#\#.fits}}\\
\noindent
{\tt{badpixel\_flat\_\#\#\#.fits}}\\
\noindent
{\tt{flat\_edge\_\#\#\#.fits}}\\
\noindent
{\tt{kmos\_wave\_band.fits}}\\ 

\noindent
\textbf{\underline{Output Files}} \\	
\noindent
{\tt{illum\_corr\_\#\#\#.fits}} - frame containing images of the internal flatfield uniformity for each IFU \\

\subsubsection{Standard Stars (Telluric Correction)}\label{subsubsec:telluric_cor}
\textbf{\underline{Recipe Name}} \\		
\noindent
{\tt{kmos\_std\_star}} \\

\noindent
\textbf{\underline{Purpose}} \\	
\noindent 
To empirically determine the atmospheric absorption as a function of wavelength and derive a conversion between electron counts and flux. On top of the input files suugested, there are several additional wavelength dependent input files described in the SPARK instructional guide which can improve the quality of the telluric correction.  \\

\noindent
\textbf{\underline{Input Files}} \\	
\noindent 
3 {\tt{STD}} type exposures of a standard star \\ 
\noindent
{\tt{illum\_corr\_\#\#\#.fits}}\\
\noindent
{\tt{master\_flat\_\#\#\#.fits}}\\
\noindent
{\tt{xcal\_\#\#\#.fits}}\\
\noindent
{\tt{ycal\_\#\#\#.fits}}\\
\noindent
{\tt{lcal\_\#\#\#.fits}}\\
\noindent
{\tt{kmos\_wave\_band.fits}}\\

\noindent
\textbf{\underline{Output Files}} \\	
\noindent
{\tt{telluric\_\#\#\#.fits}} - The derived telluric correction spectrum \\
\noindent
This recipe is basically a full science reduction, and several other reduction products are output. So far none of these have been used. 

\subsubsection{Full Science Reduction}\label{subsubsec:sci_red}
\textbf{\underline{Recipe Name}} \\		
\noindent
{\tt{kmos\_sci\_red}} \\

\noindent
\textbf{\underline{Purpose}} \\	
\noindent 
To use the above set of calibration files to reconstruct 3D object data cubes, with sky cubes subtracted. There are many options which can be supplied to this recipe, focussing on the way in which the sky subtraction is performed, and how the object cubes are combined if more than one object exposure is supplied. This is the monolithic science reduction recipe, which is comprised of several recipes which reconstruct and then combine data cubes. It is possible to use these recipes in sequence with increased configurability, but this is the simplest method to reconstruct the data cubes from each IFU. Note that the illumination correction and telluric correction files are not required for this recipe to run. \\

\noindent
\textbf{\underline{Input Files}} \\	
\noindent 
several {\tt{SCIENCE}} type object exposures, with adjacent sky exposures from the O-S-O pattern.  \\ 
\noindent
{\tt{illum\_corr\_\#\#\#.fits}}\\
\noindent
{\tt{telluric\_\#\#\#.fits}}\\
\noindent
{\tt{master\_flat\_\#\#\#.fits}}\\
\noindent
{\tt{xcal\_\#\#\#.fits}}\\
\noindent
{\tt{ycal\_\#\#\#.fits}}\\
\noindent
{\tt{lcal\_\#\#\#.fits}}\\
\noindent
{\tt{kmos\_wave\_band.fits}}\\ 
\noindent
{\tt{kmos\_oh\_spec\_\#.fits}} - reference wavelength values for the OH emission lines in the required waveband.\\ 



\noindent
\textbf{\underline{Output Files}} \\	
\noindent
{\tt{sci\_reconstructed.fits}} - set of processed and reconstructed cubes, containing 24 extensions for each IFU. One of these files is produced for every object input frame \\
\noindent
{\tt{sci\_combined\_\#\#\#.fits}} - set of combined cubes, corresponding to the data in every illuminated IFU. These files contain the grid of spectra for each object from which 1D spectra can be extracted, either as integrated object spectra or from each spatial pixel.  \\


\noindent
Note that the input files required for each recipe builds in sequence, with each recipe depending on the output of the last. Thus far this pipeline has been incredibly useful, and each recipe contains a set of configurable options to give the user more control over how the reduction is carried out. There are however some steps missing from the monolithic pipeline which can improve the quality of the resultant data and hence the accuracy of the spectra extracted from the {\tt{sci\_combined\_\#\#\#.fits}} cubes. One of my tasks has been to make these additions to the pipeline, described in section \ref{subsec:kmos_pipeline}. A further task has been to create a general and automated version of this pipeline, which combines these recipes with the additional methods described throughout section \ref{subsec:kmos_pipeline},the ultimate goal being the transformation of raw images into extracted 1D spectra at the push of a button. This automation is described in section \ref{subsec:pipeline_automation}. 




\section{Work since September 2014}\label{sec:work}
This section gives overview of the work completed since the beginning of my PhD in September. To supplement the tasks described in the following subsections, background reading for the above literature review was carried out continuously throughout this period. The notes from this reading, as well as notes from each meeting with Michele, have been recorded in an electronic journal. 

\subsection{Cosmology calculator}\label{subsec:cosmo_calc}
As an introduction to coding with Python, and to provide an important tool for the extraction of physical properties from spectra, I wrote a program to compute the age, light travel time, comoving volume, comoving radial distance, angular diameter distance and luminosity distance, $D_{L}$, at redshift z, given the set of cosmological parameters $(H_{O}, \Omega _{M}, \Omega _{V})$. This is in the same vein as Ned Wright's Cosmology Calculator \citep{Wright2006}, and makes use of the equations provided in the summary of distance measures in cosmology by David Hogg \citep{Hogg1999}. Figure \ref{fig:cosmo_calc} plots these distances in units of the Hubble distance $D_{H} = \frac{c}{H_{0}}$, against redshift for a flat cosmology with $H_{0} = 70kms^{-1}Mpc^{-1}$, $\Omega _{M} = 0.3$, $\Omega _{V} = 0.7$.  

\begin{figure}[!htp]
\centering
\includegraphics[width=0.6\textwidth]{cosmo_dist.png}
\caption{\footnotesize{\emph{Output from the cosmology calculator program, showing divergence between the different distance measures with increasing redshift. The dark blue line shows the angular diameter distance, the green line shows the luminosity distance, the red line shows the radial comoving distance and the light blue line the light travel distance.}}}
\label{fig:cosmo_calc}
\end{figure}    

\subsection{Analysing reduced SDSS spectra}\label{subsec:sdss_spec}
To motivate the development of a bespoke KMOS pipeline, we decided first to analyse the pre-reduced spectra of a set of normal, local SDSS galaxies. This back-to-front approach demonstrates the type of analysis that will be applied to KMOS galaxy spectra, albeit with significantly higher signal-to-noise than will be possible with the $z\sim 3$ KMOS galaxy sample. 
Broadly this section can be split into two categories, the first dealing with the extraction of emission line fluxes by fitting gaussians to the spectral locations of the emission lines. The second is the application of calibrated formulae to the parameters of the fitted gaussians to recover the physical properties of the galaxies. These categories are now discussed in turn.  

\subsubsection{Extracting Emission Line Fluxes}\label{subsub:ex_em}   
The reduced spectra were extracted from the `specObj' table of SDSS DR10 via SQL search. The SDSS database is a particularly useful testbed, as the optical emission line parameters as well as global galaxy properties derived from these are listed here and can be used for comparison.  The first stage of the analysis is to fit and subtract the continuum emission of the galaxy, here carried out in two separate ways. Either a third order polynomial is fit to the data, with emission lines masked, or a moving average is applied to match the continuum with a higher level of detail. Typically the simple polynomial fit provided emission line fluxes which matched the SDSS values more closely, and this is what is used throughout this section. Following continuum subtraction, it is necessary to know the redshift of the galaxy in order to determine the spectral locations of the emission line peaks. Here, and as would be the case in many other situations, the redshift of the galaxies is known and the wavelengths of the emission lines can be determined using the simple relation: 

\begin{equation}
\label{eq:redshift}
	\lambda _{peak} = (1 + z)\lambda _{vacuum}
\end{equation}

In the more general case where redshift is unknown, synthetic galaxy spectra can be used to give an indication of the expected shape of the galaxy SED. Early, intermediate and late-type synthetic spectra are plotted in Figure \ref{fig:temp_spec}. By shifting these synthetic spectra along a grid of redshift values and computing the cross correlation coefficient, $\rho$, with the observed spectrum after each shift, three separate $\rho$ arrays are determined. The maximum value of $\rho$ in each array corresponds to the redshift at which the observed and shifted template spectra best resemble each other. For these SDSS spectra, the late-type galaxy template produces by far the highest values of $\rho$ and the inferred galaxy redshift is always taken from this. An example of a cross correlation `power' plot against redshift for an observed spectrum and the shifted late-type template is shown in Figure \ref{fig:cross_cor}. The redshift value $z = 0.138$ agrees well with the value reported in the database.    \\ 




\begin{figure}[!htp]
\centering
\includegraphics[width=0.8\textwidth]{temp_spec.png}
\caption{\footnotesize{\emph{The synthetic spectra of early, intermediate and late-type galaxies, used for cross-correlation and redshift determination. Emission feautures becoming more prominent with age, as ionised nebulae become enriched by successive generations of star formation.}}}
\label{fig:temp_spec}
\end{figure} 

\begin{figure}[!htp]
\centering
\includegraphics[width=0.8\textwidth]{Cross_corr.png}
\caption{\footnotesize{\emph{The cross correlation `power' is plotted against the grid of redshift values passed to the function, showing a clear peak at $z\sim 0.138$. This is a relatively easy procedure for such high quality spectra with prominent emission lines, becoming increasingly difficult to gain enough signal as redshift increases. }}}
\label{fig:cross_cor}
\end{figure} 

Once the redshift of the galaxy has been determined, equation 4 gives the recipe for finding the wavelengths of the peaks of the emission lines. A table of the known vacuum wavelengths for H$\beta$, [OIII]$\lambda 4959$, [OIII]$\lambda 5007$, H$\alpha$, [NII]$\lambda 6585$, [SII]$\lambda 6718$ and [SII]$\lambda 6732$ was used to find the observed wavelengths for each galaxy, and a gaussian function is fit to the data surrounding these emission line peaks in turn by fixing the central value. The area underneath the gaussian is the flux of that emission line and is also used to find the line equivalent width, whereas the width of the gaussian is used to determine the FWHM of the line, important for velocity measurements. A plot is made of the whole spectrum, with each emission line overlaid in red, to check the quality of these fits. Figure \ref{fig:fitted_spec} (a) gives an example of this, with panel (b) showing the wavelength region surrounding $H\alpha$ and confirming that the fitted gaussians closely match the shape of the emission lines.\\    

\begin{figure}[!htp]
\centering
\subfloat[]{\includegraphics[width=0.5\textwidth]{fitted_spectrum_full.png}}
\subfloat[]{\includegraphics[width=0.5\textwidth]{fitted_spectrum_part.png}}
\caption{\footnotesize{\emph{An example of one of the SDSS spectra for a $z\sim 0.14$ galaxy. Panel (a) shows the galaxy spectrum in blue, with the fitted gaussians overlaid in red. Panel (b) shows the same with an expanded view of the region $7400-7700$\AA, encompassing the $H\alpha$, $[NII]\lambda 6585$, $[SII]\lambda 6718$ and $[SII]\lambda 6732$ emission lines, shifted in wavelength by the factor $(1 + z)$. Plot (b) clearly demonstrates the high quality of the fit.}}}
\label{fig:fitted_spec}
\end{figure}



\subsubsection{Recovering Physical Properties}
After recovering line fluxes from the galaxy spectra, these can be converted to line luminosities using the luminosity distance, $D_{L}$, to the galaxy at redshift z computed with the cosmology calculator. For example, the luminosity of H$\alpha$ at redshift z, recovered from flux F($H\alpha$) is given by: 

\begin{equation}
\label{eq:flux-lum}
 	L(H\alpha) = 4\pi D_{L}^{2} F(H\alpha)
\end{equation} 

As mentioned in section \ref{subsubsec:em_lines}, H$\alpha$ is a particularly sensitive tracer of the instantaneous star formation rate, as only the most massive, and hence short lived, stars contribute significantly to the ionising photon flux. Using the relation provided in Kennicutt and Evans \citep{Kennicutt_2012}: 

\begin{equation}
 	\label{eq:halpha_sfr}
 	SFR(H\alpha) = log_{10}(L(H\alpha)) - 41.27
 \end{equation} 

an estimation of the SFR in each galaxy is be found. Also, using the PP04 \citep{Pettini_2004} and KD02 \citep{Kewley2002} abundance indicators (Equations 7 and 8 respectively), two separate estimations of the metal abundance are found. The PP04 indicator is found to agree relatively well with the SDSS oxygen abundance, but as described in \ref{subsec:abundance} there are discrepancies between the abundances reported from different indicators.  

\begin{equation}
	\label{eq:PP04}
	12 + log(O/H) = 9.12 + (0.73\frac{F([NII]\lambda 6585)}{F(H\alpha)})
\end{equation}

\begin{equation}
	\label{eq:KD02}
	12 + log(O/H) = 8.73 - (0.32\frac{F([OIII]\lambda 5007)F(H\alpha)}{F([NII]\lambda 6585)F(H\beta}))	
\end{equation}

As a final test of the accuracy of the emission line fluxes extracted from the SDSS spectra, a BPT diagram was plotted for a sample of 50,000 test galaxies in Figure \ref{fig:owen_bpt}. These data are a subset of the grey points in Figure \ref{fig:steidel_bpt} and the distributions closely resemble one another. The subset of galaxies with higher ionisation parameter are clearly seen as those with higher $\frac{[OIII]}{H\beta}$ for given $\frac{[NII]}{H\alpha}$.


\begin{figure}[!htp]
\centering
\includegraphics[width=0.5\textwidth]{bpt_owen.png}
\caption{\footnotesize{\emph{The BPT AGN diagnostic diagram recovered from applying the gaussian fitting procedure described above for a sample of 50,000 galaxies extracted from the specPhoto table of SDSS DR10. There is clear bimodality in the diagram, reflecting the ionisation state of galaxies with and without AGN.}}}
\label{fig:owen_bpt}
\end{figure}    

It would have been interesting to investigate stellar masses recovered from SPS modelling and recover the local MZ-relationship as well as the local SFR vs. $M_{*}$ curve, but we decided not to pursue stellar masses at this stage as this was more an introduction to the topic than a rigorous analysis. \\ 

It is likely that the  spectrum fitting strategy will have to be significantly aletered when moving to higher redshifts, fainter galaxies and lower S/N, but this was helpful for learning the general principles involved in extracting fluxes and galaxy physical properties from pre-reduced spectra. We finished this section towards the middle of December, turning attention to KMOS and the reduction pipeline at the end of December and after the Christmas break.   

\subsection{Additions to the KMOS reduction pipeline}\label{subsec:kmos_pipeline}

As described in section \ref{subsec:kmos_pipeline}, following the recipes supplied in the ESO reduction pipeline allows the user to process raw object images into fully reduced data cubes. However, the `out of the box' pipeline skips several steps which can greatly increase the quality of the reconstructed data cubes. The importance of these steps increases during the reduction of high redshift galaxy images, where the signal to noise ratio is significantly lower. It is important to tease as much information out from the data as possible at these redshifts, especially when measuring the strengths of nebular emission lines. Ground based near-IR spectroscopy is plagued by emission from the OH molecule, and so to extract reliable information from galaxy images, sky-subtraction must be performed as accurately as possible. The additions made to the ESO pipeline focus on improving the sky-subtraction process prior to science reconstruction, and correcting the infrared detectors for read noise effects. These additions are decsribed below.  \\ 
The overarching goal at this stage is to produce an automated pipeline which reliably extracts as much information as possible from the KMOS high redshift galaxy images. 

\subsubsection{Read noise}\label{subsubsec:read_noise}

The KMOS infrared detectors are read out in columns of 64 pixels, with a seperate amplifier for each column to boost the recorded signal. Each amplifier behaves differently to its neighbour, and the effect produced by one amplifier is not constant with time, leading to a banded flux bias effect across the sky subtracted object image, as shown in the top panel of Figure \ref{fig:column_sub}. This is carried through the data reduction pipeline, resulting in excess bands of flux across spatial pixels in the reconstructed object images. These problems reflect the current limitations of IR detectors and the difficulty of converting photon energies into electronic signals in this region of the spectrum. There are two different methods to compensate for this read bias, both involving a correction to the raw object image prior to sky subtraction. The image in Figure \ref{fig:column_sub} is ideal for testing this effect; the chosen detector is illuminated by the light from only 4 IFUs. This leaves a large `test-pixel section' not contaminated by sky/object flux in the centre of the detector for evaluating the quality of corrections made. The light blue curve in Figure \ref{fig:med_corr} shows the uniformity of spatial pixels prior to making any column-corretions, plotting the median of 100 spectral pixels for each spatial pixel in the test pixel section. Clearly the median jumps around, changing with each 64 or 128 pixel segment. \\ 

\begin{figure}[!htp]
\centering
\includegraphics[width=0.8\textwidth]{bad_column.png}
\includegraphics[width=0.8\textwidth]{good_column.png}
\caption{\footnotesize{\emph{The top panel shows a raw sky subtracted object image, whereas the bottom shows the same subtraction after the object image has been corrected for readout bias using the segments method. There is a significant improvement in the uniformity of the pixels after corretion. Both images show an 800x2048 section of one of the near-IR detectors, with 4 IFUs operational (these being the significantly brighter pixel columns).}}}
\label{fig:column_sub}
\end{figure}

The first correction method uses the top four spectral pixels of a raw sky subtracted object frame, these being masked `reference' pixels which are never illuminated by light from the IFUs. The correction is made by taking the median of these top four pixels in columns of 64, and subtracting this from each 64 pixel column of the object image prior to sky image subtraction. This accounts for pixel uniformity across the image, but still leaves the median value of the test pixel section offset from zero. The red curve of Figure \ref{fig:med_corr} shows the performance of this method, clearly improving the spatial uniformity of the uncorrected test pixel section.  \\ 

The second correction method uses the wavelength calibration `lcal' frame, output from the reduction pipeline. This frame has the same dimensions as the detector, contains `nan' values for pixels not illuminated by an IFU and the wavelength values of spectral pixels which are. There is a four pixel non-illuminated gap between each illuminated slitlet, and so by looking at pixels on the sky subtracted image corresponding to nan locations on the lcal frame, a column correction can be found.For each 64 pixel spatial column, the median value of the non-illuminated pixels of the subtracted image is found, and depending on whether this is negative or positive, it is either added or subtracted to the raw object image. A third correction method which uses a variation of this technique is also applied; instead of finding the correction in each 64x2048 column, the spectral axis is split into 128 pixel segments, and 16 64x128 corrections are found in each column. This caters for systematic variations in brightness across a detector not directly caused by readout bias. These two variations will be referred to as the lcal and lcal-segments correction methods. The dark blue and green curves of Figure \ref{fig:med_corr} respectively plot the performance of these two methods in the test pixel section. For the chosen frame the performance is equivalent, with the dark blue curve underneath the green, and the median pixel value shifted to zero. For further comparison between the raw and corrected sky subtracted frame, the counts per pixel in the test-pixel section were binned and a gaussian fit to the resultant histogram. The green and blue curves in Figure \ref{fig:pix_hist} show the pre and post-correction values respectively and demonstrate the shift of the median pixel value towards zero when the correction is applied.    \\       

Due to the lcal-segments method automatically shifting the median value of the test pixel segment to zero, this was chosen as the bias correction method for the pipeline. Since the bias effect varies between the detectors and the frames, the correction is applied in a script which passes the lcal-segments method all of the object-sky pairs in sequence. The correction is applied to each object frame, saving these with a `$\_$Corrected' extension.  


\begin{figure}[!htp]
\centering
\includegraphics[width=0.8\textwidth]{medians3.png}
\caption{\footnotesize{\emph{The performance of the three different column correction methods are compared against the median from the uncorrected sky subtracted object image. The light blue curve shows the uncorrected median, which jumps around in bands of 64 pixels, the red shows the top four method and both the dark blue and green curves lie on top of one another, demonstrating the lcal and lcal-segments methods respectively. The spatial uniformity of the pixels is greatly improved by all of the methods, and the two lcal methods also shift the median of the test pixel segment to zero.}}}
\label{fig:med_corr}
\end{figure}


\begin{figure}[!htp]
\centering
\includegraphics[width=0.5\textwidth]{pix_hist.png}
\caption{\footnotesize{\emph{To quantify the improvement made to the corrected subtracted image, a slice of pixels in between illuminated IFUs in both the raw and corrected subtracted images of Figure \ref{fig:column_sub} was extracted. The pixel values were placed in bins and a gaussian fit to the resultant histogram, blue for post-correction and green for pre-correction. The correction has the effect of shifting the median subtracted pixel value to zero, which is what we want for a clean sky subtraction. }}}
\label{fig:pix_hist}
\end{figure}





\subsubsection{OH Emission and Sky Subtraction}\label{subsubsec:Oh_lines}

Emission lines produced by the OH radical dominate near-IR spectra, appearing between 0.61 - 2.62 $\mu m$ and with fluxes several orders of magnitude larger than other sky emission. These are produced by Meinel rotation-vibration transitions of the OH molecule \citep{Meinel1950}, and appear in distinct bandings across the near-IR spectral range, as shown in Figure \ref{fig:sky_spec}. As a result, the removal of OH lines is an essential stage in the processing of near-IR data, with images of blank patches of sky being taken and then subtracted from the `object' images. Waves in the upper atmosphere cause a strong time dependence in both the absolute and relative intensities of these emission lines, complicating the removal process. For a clean subtraction, the flux of the OH lines in the sky and object images would have to vary by much less than 1 percent; yet the lines can vary significantly on time-scales of only a few minutes and often exposures times longer than this are required to achieve adequate S/N. To further complicate the subtraction process, manufacturing variations between instrument elements, and even small amounts of spectral flexure in the instrument during rotation can leave significant `P-Cygni' shaped residuals in the spectrum when a sky frame is subtracted. \\ 
So one cannot simply take the sky spectrum in one spatial pixel and subtract it from another, and for the reasons above, despite the rotation-vibration transitions of OH being very well understood \citep{Osterbrock1996}, attempts to generate and subtract synthetic OH spectra inevitably fail due to instrumental limitations and the time variation of OH emission. On a positive note, these lines serve as useful reference points for wavelength calibration of astronomical spectra, and near-IR sky emission line maps have been created to aid in the identification of these lines \citep{Rousselot2000}. Generally the flux variations of OH lines are not a problem for long-slit spectra, since the slit tends to be much more extended than the source. The OH lines can then be completely removed from the rectified two-dimensional spectrum by fitting a function to the background at each spectral pixel. For near-IR IFUs this cannot be done, since the field of view is much smaller and astronomical targets will occupy almost all of this, forcing the observation of separate blank sky fields.\\ 


\begin{figure}[!htp]
\centering
\includegraphics[width=0.8\textwidth]{sky_spectrum.png}
\caption{\footnotesize{\emph{A plot of the OH emission dominating the 1-1.4$\mu m$ region of the spectrum, extracted from a YJ-band KMOS skycube. The rotation-vibration bands of the OH molecule are clearly seen. These are extremely strong features, peaking at almost 5000 counts $s^{-1}$, orders of magnitude above the object flux.}}}
\label{fig:sky_spec}
\end{figure}

Davies \citep{Davies2007} describes a powerful method for subtracting these OH emission lines from reconstructed object datacubes. To a large extent, transitions between any two vibrational bands of the OH molecule lie between well defined wavelength limits. By splitting extracted spectra into these wavelength ranges and looking at the spectral pixels that contain bright OH lines in both the object and sky cubes, a wavelength dependent optimal scaling between the object and sky flux can be found. This is applied to the object cube to enable the best possible match between the amplitudes of the OH lines in object-sky pairs. A step by step guide to this procedure is given in the paper by Davies \citep{Davies2007}, and the essence of the algorithm is included optionally in the KMOS science reduction pipeline \footnote{To include optimal scaling of the skylines, the --skytweak=TRUE argument is supplied to the kmo\_sci\_red recipe.}. Applying the above scaling does not however account for the spectral shifts between object-sky pairs caused by the rotation of the instrument. There is an intermediate step in the Davies algorithm which shifts the object cube along the wavelength axis to best align OH lines. We question whether it is more appropriate to determine any spectral and spatial shifts from the raw images, prior to cube reconstruction. The following is a description of the algorithm I wrote to compute and apply these shifts to the raw images, before they are fed to the cube reconstruction recipes. \\ 

Panel (a) of Figure \ref{fig:pre_shift} shows an example of a distinct set of P-Cygni features in a raw sky subtracted object image. These are the bandings of white and black horizontal lines which are clearly separated from one another, indicative of a misalignment between the raw object and sky images used for the subtraction. The goal is to find the optimal spatial ($\Delta x$) and spectral ($\Delta y$) pixel shifts which can be applied to the object frame to minimise this effect. The effect is said to be minimised when the correlation between the object and sky images is a maximum. This correlation is quantified by evaluating the cross-correlation coefficient, $\rho _{frames}$, between the object and sky frames using all `reliable' pixels on the 2D detector, given in Equation 9. $O_{xy}$, $S_{xy}$ are the individual object and sky pixels respectively, and $\overline{O}$, $\overline{S}$ are the mean values of the chosen pixels in the object and sky frames. 

\begin{figure}[!htp]
\centering
\includegraphics[width=0.4\textwidth]{pre_shift.png}
\includegraphics[width=0.4\textwidth]{post_shift.png}
\caption{\footnotesize{\emph{Panel (a) shows the distinct white/black pattern of the OH lines in the subtracted image before correcting the object. This effect would carry through to the extracted 1D spectrum, leaving a P-Cygni residual shape. Panel (b) shows the improvement made after applying the optimal shift. The subtraction leaves negative values in the place of the OH lines, which are then scaled up by the Davies skytweak algorithm.}}}
\label{fig:pre_shift}
\end{figure}   

\begin{equation}
	\label{rho_frames}
	\rho_{frames} = \sum\limits_{x,y} \frac{(O_{xy} - \overline{O})(S_{xy} - \overline{S})}{\sqrt{(O_{xy} - \overline{O})^{2}(S_{xy} - \overline{S})^{2}}}
\end{equation}



Reliable pixels are those above a certain flux level indicative of real signal, those below an upper flux limit which would swamp the correlation coefficient and those which remain after masking the bad pixels returned by the KMOS calibration recipes for the same reason. In practice, we found that far better results are found when the bad pixel mask is `grown'. Although the recipes reveal the worst pixels, often the areas surrounding these are hot and contribute heavily to $\rho_{frames}$. Growing the mask refers to masking off an additional cross of pixels, one above, below, to the left and to the right, for every bad pixel identified in the calibrations. In short, the `reliable' pixels which remain are dominated by flux from the OH lines, which are exactly the structures we wish to correlate. The shift must be applied to the object due to the KMOS O-S-O observing strategy; each sky frame can be paired with two object frames and the shifts will be different each time. We expect that the shifts will be of order 0.1 pixels in both directions. Initially a grid of ($\Delta x$, $\Delta y$) pairs were constructed, ranging from $\Delta x, \Delta y = -0.5,-0.475,...0.5$ with 0.025 pixel increments and every value of $\Delta y$ explored for each $\Delta x$. Each pair is fed into the Pyraf `imshift' routine, which fits a function to the pixel surface and interpolates each new pixel value at the shift locations. $\rho _{frames}$ is computed for every shift, and the maximum value is chosen as indicator for the ($\Delta x, \Delta y$) pair to apply to each detector. Figure \ref{fig:column_sub} (b) shows the same column of pixels as in (a), with the optimal shift applied to the object prior to subtraction. There is a marked improvement over the original bandings, suggesting that the procedure has increased the alignment between object and sky frames. Figure \ref{fig:rho_plot} shows contours of the surface of $\rho _{frames}$ values computed over the grid passed to imshift. There is a clear, global maximum value at roughly ($\Delta x = 0.025, \Delta y = 0.055$) suggesting that the interpolant is well behaved. \\


       
\begin{figure}[!htp]
\centering
\includegraphics[width=0.7\textwidth]{KMOS_SPEC_OBS258_0009_Corrected110_temp_spline3_CorrelationGraph.png}
\caption{\footnotesize{\emph{A contour plot for $\rho$ computed for the grid of x, y shift values passed to Pyraf imshift. The contours are centred on the maximum value of $\rho$, this corresponding to the shift values which best align the object and sky images. The contours are drawn at $\rho _{max} - 0.25\sigma$, $\rho _{max} - 0.75\sigma$ and $\rho _{max} - 1.5\sigma$ and indicate the presence of a global maximum, suggesting unique `best' shift values $\Delta x$ and $\Delta y$.}}}
\label{fig:rho_plot}
\end{figure}

To check whether the recovered spectral shifts change as a function of position on each detector, the routine allows for each detector to be split into an arbitrary number of vertical and horizontal sections. The above process is then applied to each section in turn before recombining the shifted sections into the full detector. Generally we found that spatial variation in the shift is small, and that this starts to become a case of diminishing returns when the time required to apply the routine to all sections is taken into account. There are three detectors in every object frame, each accommodating eight IFUs, and in an observing block many frames will be collected to stack together. It is time consuming to pass the full grid of $(\Delta x, \Delta y)$ pairs to imshift, especially when segmenting each detector, and so to speed up the routine a downhill simplex algorithm is used. Values of $(\Delta x, \Delta y)$ which lead to increasing $\rho_{frame}$ evaluations are explored until a given precision is reached. This is a valid approach to take after verifying that there is a single $\rho_{frame}$ maximum, and greatly decreases the time taken to run the shift routine. In a similar way to the column correction method, the shift routine is applied to each object-sky pair in sequence, saving the shifted object with an extension indicating the number of horizontal and vertical segments each detector has been split into, e.g. ...\_11\_Shifted.fits. Figure \ref{fig:shift_plot} shows the values of ($\Delta x, \Delta y$) pairs for a series of frames, also plotted with instrument rotator angle to look for correlation. Clearly there is an oscillatory structure to the spectral shift, this reflecting the fact that the same shift is applied to each of the three detectors within each science frame. \\ 


\begin{figure}[!htp]
\centering
\includegraphics[width=\textwidth]{shift_plot.png}
\caption{\footnotesize{\emph{The top panel shows the rotator angle, which shifts to its new value after every three detector ID's (single frame). The middle panel shows the disordered best shift $\Delta x$ values. The bottom panel shows the oscillatory structure recovered in the best $\Delta y$ values. This arises as a result of }}}
\label{fig:shift_plot}
\end{figure}

As an illustration of the improvement made by applying the shift, we examine the spectra of red-supergiant stars (extracted using the method described in section \ref{subsubsec:optimal_spec_extract}) from datacubes reconstructed in different ways. First, no shift is applied to the column corrected object frame before feeding through the reduction recipe, and the skytweak algorithm is not applied. The top panel of Figure \ref{fig:p_cygni} shows the spectrum extracted from one of the IFUs in the resultant datacube, plotted between 1.15-1.20$\mu m$ of the KMOS YJ-band, and with corresponding sky spectrum underneath. It is important to note the vastly different flux scales between the object and sky spectra, highlighting the importance of optimising the sky subtraction process. Clearly the spikes in the object spectrum correspond to the OH emission lines, and in this case show the P-Cygni profiles expected from applying no shift. Second, the shift routine is applied to the column corrected object frames and these are fed into the reduction recipe without skytweak applied. The middle panel of Figure \ref{fig:p_cygni} demonstrates the results of this, effectively eliminating the P-Cygni profile, but with the sky lines still in need of scaling. Third, the same is done as in the second analysis but with the skytweak option applied. The bottom panel of Figure \ref{fig:p_cygni} shows that the skylines are almost all successfully scaled away. This leaves the question of whether it is better to apply the shift to the raw data frames, or to the datacubes as part of the Davies algorithm. To test this, the column corrected object images were fed through the reduction pipeline with skytweak applied, allowing the algorithm to compute the spectral and spatial shifts from the skycube. A very similar plot to the bottom panel of Figure \ref{fig:p_cygni} is recovered, and a more detailed inspection is required to shed light on this question, a topic which is discussed in section \ref{frame_diagnostics}.


\begin{figure}[!htp]
\centering
\includegraphics[width=0.8\textwidth]{spec_compare_cygni.png}
\includegraphics[width=0.8\textwidth]{spec_compare.png}
\includegraphics[width=0.8\textwidth]{spec_compare_tweak.png}
\caption{\footnotesize{\emph{}}}
\label{fig:p_cygni}
\end{figure}

\subsubsection{Frame Diagnostics}\label{frame_diagnostics}
The quality of the sky subtraction can vary significantly on a frame by frame basis, and so when stacking many reconstructed datacubes to increase S/N for an object, one can carry through the results of poor sky subtractions to the whole stack. It is necessary to check the quality of sky subtraction on a frame by frame basis, by feeding a list of object/sky pairs through the science reconstruction recipe. The spectral locations of strong OH emission lines are found by examining a reconstructed skycube, and selecting all spectral pixels where the flux exceeds 500 counts. The quality of the sky subtraction is found for each IFU by optimally extracting the object spectrum following the prescription described in section \ref{subsubsec:optimal_spec_extract}, taking the median of the object flux at the spectral locations identified from the skycube and normalising this number by the object continuum. The object continuum is defined as the median of the object flux at all spectral locations that are NOT flagged as OH lines from the skycube. This process provides a sky performance statistic; one would expect that if the sky subtraction is performed well, the statistic would be close to 1. Figure \ref{fig:sky_sub_perf} shows a summary of this statistic for the red-supergiant stars. The left panel shows the average sky subtraction performance across all IFUs for each science frame and the right panel plots for each frame the sky subtraction performance as a function of IFU. The most detailed view of the sky subtraction process is given in Figure \ref{fig:ifu_subplots}, which drills down to subplots of the sky subtraction performance in each IFU as a fuction of IFU frame. Note that the IFUs are arranged in rows of 8 to reflect the three detectors, and some of the IFUs were not operational during the exposures. There are two curves in Figure \ref{fig:ifu_subplots}; the blue curve represents the quality of the sky subtraction when the spectral shift is computed from the raw images using my algorithm; the red curve shows the quality when the shift is computed from the datacubes with the Davies algorithm. Clearly the performance is comparable, and sometimes the quality is better using each of the two methods. Currently due to the raw frame shift being much more time intensive, the Davies algorithm applied on its own is favourable. \\ 


From a combination of the plots in Figures \ref{fig:sky_sub_perf} and \ref{fig:ifu_subplots} it is easy to determine where the sky subtraction isn't performed as well, and hence which frames to omit from the stack. This is done automatically, with all frames where the average sky subtraction statistic is less than 0.5 or greater than 1.5 being discarded. From both of these figures it is clear that the sky subtraction is significantly worse for some IFUs than others, and that some frames have much worse sky subtraction performance overall. The reason for these spurious frames and IFUs still isn't clear, and it is an ongoing process for us to understand why the sky subtraction process should sometimes be performed much worse than others. One explanation is that some of the sky IFUs in the observed data are contaminated by being placed on a patch of sky which is not blank. 


\begin{figure}[!htp]
\centering
\subfloat[]{\includegraphics[width=0.48\textwidth]{frame_performance_double.png}}
\subfloat[]{\includegraphics[width=0.48\textwidth]{IFU_by_frame.png}}
\caption{\footnotesize{\emph{The wavelengths of prominent OH emission lines are determined by looking at the spectrum of a skycube. A sky subtraction performance `statistic' is found by taking the median of the object cube flux at these wavelengths,. The left panel shows the quality of the sky subtraction as a function of frame, taking an average of the statistic across all IFUs in the frame. The right panel shows the statistic as a function of IFU, plotted for each of the frames taken.}}}
\label{fig:sky_sub_perf}
\end{figure}

\begin{figure}[!htp]
\centering
\includegraphics[width=0.7\textwidth]{IFU_subplots_double.png}
\caption{\footnotesize{\emph{The sky subtraction performance is plotted in IFU as a function of frame. The blank plots are those IFUs which were not operational when the data were collected. This plot has the potential to highlight an optimal combination of frames to stack for each object. The blue curve represents the performance of the sky subtraction when the spectral shift is computed from the raw images and the red curve when the shift is computed from the datacubes with the Davies algorithm. It is difficult to distinguish in terms of performance a consistently `better' method, however the Davies algorithm is significantly faster to run. Some of the IFUs show a flat curve close to 1 across all frames; others jump around significantly and are much higher than 1 at all points.}}}
\label{fig:ifu_subplots}
\end{figure}

As a further frame-by-frame inspection, we check the PSF of a standard star for each of the frames. During an OB, KMOS dedicates one of the IFUs to observing a standard star, to monitor the variation in its PSF and hence variations in seeing as the night progresses. For the red-supergiants, a single bright object is chosen for this analysis. A 2D gaussian is fit to the object image and the FWHM of this gaussian gives an indication of the instantaneous seeing at the time the frame was observed. Figure \ref{fig:fwhm_track} shows the evolution of the FWHM of the standard star on the same night as the data in Figure \ref{fig:sky_sub_perf} were taken.

\begin{figure}[!htp]
\centering
\includegraphics[width=0.5\textwidth]{frame_fwhm.png}
\caption{\footnotesize{\emph{A standard star is tracked by one of the KMOS IFUs throughout the duration of the OB. This gives an indication of the seeing as each of the object frames are observed. The plot shows the evolution of the seeing throughout the course of the night, by tracking a bright star in one of the IFUs over all the frames, fitting a 2D gaussian to the flux, extracting the properties of the gaussian and converting to seeing using the pixel scale recorded in the fits header.}}}
\label{fig:fwhm_track}
\end{figure}

From the recovered FWHM of the 2D gaussian, the objects are divided into four bins of seeing quality. The `Best' bin has seeing better than 0.6$^{\prime\prime}$; the `Good' bin has seeing between 0.6$^{\prime\prime}$ and 1$^{\prime\prime}$. The `Okay' bin has seeing between 1$^{\prime\prime}$ and 1.5$^{\prime\prime}$; the `Bad' bin has anything worse than this, and typically any objects falling within this category are discarded. The objects in each of the four bins are then treated independently for optimal spectrum extraction as described in section \ref{subsubsec:optimal_spec_extract}. The motivation behind this procedure is that it would be unwise to degrade the quality of an image with good seeing, by stacking together in equal proportion with an object with bad seeing. The above analysis gives control over how stacking is carried out and which weights should be applied to the different seeing bins. In sum, objects are categorised into these four bins depending on sky performance statistic and seeing quality, both of which affect the resultant 1D object spectrum.  


\subsubsection{Optimal Integrated Spectrum Extraction}\label{subsubsec:optimal_spec_extract}

To increase S/N, it is necessary to extract the spectrum from a galaxy with weightings corresponding to where most of the flux is being emitted. Galaxies have a given radial profile, and light from each part of this profile is spread out into a seeing disc when it hits the atmosphere. On the other hand, stars are point-like, and the seeing disc from a star originates from a single point-source of light. By tracking a star in one of the IFUs, we get a measure of the theoretical minimum seeing and can recover the set of weights which can be applied to each spatial pixel during the spectrum extraction. To expand on this, consider Figure \ref{fig:fwhm_image}. A 2D gaussian is fitted to the image of the tracked star in a particular object stack, recovering the parameters of the gaussian function. By normalising the gaussian to unity, and truncating the function at a given number of standard deviations from the centre, one recovers a weighting mask for spectrum extraction which can be applied to all objects in that stack by shifting the mask centre. The resultant 1D spectrum is then recovered by summing across all weighted spatial pixels. This procedure is applied to all of the FWHM bins described in the above section, to recover a set of `Best', `Good' and `Okay' 1D spectra for each of the objects observed by KMOS. Following this, an analysis similar to that described in section \ref{subsub:ex_em} would be applied to recover physical properties from the emission features in the 1D spectrum. 


\begin{figure}[!htp]
\centering
\includegraphics[width=0.5\textwidth]{gauss_img.png}
\caption{\footnotesize{\emph{An example of one of the 2D gaussian fits to the stacked `tracked' star in the `Best' FWHM bin. From this fit the profile for spectrum extraction is recovered.}}}
\label{fig:fwhm_image}
\end{figure}	


\subsection{Pipeline Automation}\label{subsec:pipeline_automation}
As mentioned towards the end of section \ref{sec:KMOS}, the pipeline becomes far more usable if the amount of interaction required is minimised. To use the recipes described in section \ref{sec:reduction}, one would have to construct the appropriate SOF text file after executing each recipe, taking care to be consistent with the waveband used and the exposure times of the individual files. Constructing these takes time, interaction between each of the stages and errors can be made. I have coded an automated version of the full reduction pipeline, which sequentially executes each of the recipes described in section \ref{sec:reduction}, taking information from the fits headers of the raw and calibration files to hierarchically build the appropriate SOF text files for each recipe. The pipeline additions described throughout sections \ref{subsubsec:read_noise} - \ref{subsubsec:optimal_spec_extract} are included in this automated pipeline, with the end result being a Science directory populated with subfolders containing the optimally extracted 1D spectra for each of the seeing bins.
\subsubsection{Automated Pipeline Usage}
Currently the usage of this pipeline is as follows. The user must have the following environment variables specified: \\ 

\noindent
{\tt{\$KMOS\_CALIB}} - Pointing to the static calibration directory (doesn't change) \\
\noindent
{\tt{\$KMOS\_SCIENCE}} - Pointing to the current science output products directory \\
\noindent
{\tt{\$KMOS\_RAW}} - Pointing to the raw data directory \\
\noindent
{\tt{\$KMOS\_DYN\_CAL}} - Pointing to the calibration directory for the pipeline run \\

\noindent
Accompanying these environment variables the user must have created the corresponding SCIENCE, RAW and CALIBRATIONS folders. Currently there must be a separate directory structure for each waveband, with the RAW folder containing the calibration and object files for that waveband alone. Initially the calibration and science directories should be, but do not need to be, empty. The user executes the python pipeline script from the calibrations directory, the steps are executed sequentially and the SCIENCE directory is populated with the reduction products for that waveband. \\
In the future, I plan to include the functionality to cope with multiple wavebands and multiple OBs in a single RAW folder, as currently there is still user input required to arrange the wavebands into different raw folders prior to clicking go.   



\section{Planned projects}\label{sec:projects}
In sections \ref{subsec:redshift_three_sample} - \ref{subsec:LAEs}, upcoming projects involving the developed KMOS pipeline are briefly discussed. The main publication prospect comes from section \ref{subsec:redshift_three_sample}, for which some of the data have already been taken, and will be complemented by a September observing run to Paranal. 

\subsection{The physical properties of galaxies at z = 3}\label{subsec:redshift_three_sample}
This project is similar to that of Troncoso et al. \citep{Troncoso_2014}, in which the authors studied the spatial gradients in metallicity, SFRs and velocity across the face of a sample of $\sim 30$ galaxies with $<z> = 3.4$. The selection of these galaxies was biased towards the brightest objects, and is hindered by small number statistics. The KMOS sample will consist of $\sim 80$ objects which sample a larger fraction of the luminosity function. The spatially resolved physical properties of these galaxies will be explored, and there is potential to examine both integrated and spatially resolved constructions of the fundamental metallicty relationship. Particular attention will be given to the spatial gradients of chemical abundance throughout this sample, and whether any correlation with the kinematic properties of the galaxies can be uncovered. Using HST data in the KMOS fields, the morphological properties of the galaxy sample can be explored, again searching for correlations with other physical properties and for the presence of mergers. \\ 
As mentioned above, the main aim here is to look at gas enrichment and kinematics and doing this in a spatially resolved fashion for a large sample of galaxies at this redshift is novel.     




\subsection{Searching for CIII and CIV Emission}\label{subsec:CIV_emission}
Only 1 in every 100 galaxies at locally showing signs of CIII or CIV emission in their spectra, suggesting that it is a rare occurence for the radiation field to be strong enough to either doubly or triply ionise carbon. The promising detection of CIV emission at $z \sim 7$ described by Stark \citep{Stark2015} in a sample of only four galaxies suggests that CIII/IV emission could be more ubiquitous at high redshift, in turn suggesting a harder ionising radiation field. At these redshifts there is little prospect of detecting the resonant emission line Lyman$\alpha$, and indeed the detection of UV rest frame metal lines may be the only hope of progress for spectroscopic confirmation. Deep KMOS data at $z\sim 5-6$ can be scoured for these emission lines, which may yield spectroscopic confirmations of galaxies within this range and provide information about the excitation conditions in high redshift. 



\subsection{Spectroscopic Confirmation of LAEs at z=7-8}\label{subsec:LAEs}
A still higher redshift KMOS sample can be searched for the presence of Lyman-$\alpha$ emission, in a similar approach to that adopted by e.g. Finkelstein et al. \citep{Finkelstein2013}. This is a rather tentative prospect, as Lyman-$\alpha$ detections have proved to be extremely rare in this epoch of re-ionisation, where Lyman-$\alpha$ photons would be instantly absorbed by neutral hydrogen. Nonetheless, the KMOS YJ band would be sensitive to a Lyman-$\alpha$ detection at these redshifts, and if the data is there, why not search it!

\section{Summary}
To briefly summarise, between September 2014 and May 2015 I have: 
\begin{itemize}
	\item Reviewed literature relevant to the project areas, mainly focussing around the measurement of galaxy physical properties from spectroscopic data supported by ancillary multi-wavelength photometry
	\item Studied the KMOS instrument, data analysis pipeline and science objectives in detail 
	\item Developed some technical machinery for the extraction of fluxes from reduced 1D SDSS spectra 
	\item Developed additions to the KMOS pipeline to improve the quality of the resultant 1D spectra extracted fro reconstructed datacubes 
	\item Automated the KMOS pipeline to go between raw images taken through any filter/grating and extracted 1D spectra, taking into account seeing conditions and the quality of OH subtraction 
	\item Discussed with my supervisor and secondary supervisors the plan moving forward, which is to use this pipeline on a set of $z\sim 3$ galaxies and explore the spatial gradients in their physical properties 
\end{itemize}
%%%%%%%%%%%%%%%%%%%%%%%%%%%%%%%%%%%%%%%%%%%%%%%%%%%%%%%%%%%%%%%%%%%%%%%%%%%%%%%%%%%%%%%%%%%%%%%%%%%%%%%%%%%%%%%%%%%%%%%%

\clearpage 
\bibliographystyle{apj.bst}
%\bibliography{/usr/local/texlive/texmf-local/bibtex/bib/ojt.bib}
\bibliography{/Users/owenturner/Documents/PhD/KMOS/Latex/Bibtex/library.bib}

\end{document}
